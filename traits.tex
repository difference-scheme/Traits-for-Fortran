\documentclass[11pt,oneside]{report}
\usepackage[a4paper, total={6.5in, 9in}]{geometry}
\usepackage{natbib}
\usepackage{url}
\usepackage{listings}
\usepackage{listings-rust}
\usepackage[scaled=0.82]{beramono}
\usepackage[T1]{fontenc}
\usepackage{hyperref}
\usepackage{xcolor}
\usepackage[title]{appendix}
\usepackage{csquotes}
\usepackage{mathpazo}
%\usepackage{mathptmx}
\usepackage{array}
\usepackage{booktabs}
\usepackage{tablefootnote}

\newcolumntype{R}[1]{>{\raggedleft\arraybackslash}p{#1}} 
\newcolumntype{L}[1]{>{\raggedright\arraybackslash}p{#1}} 
\newcolumntype{C}[1]{>{\centering\arraybackslash}p{#1}}

\newcommand{\code}[1]{{\selectfont\ttfamily{#1}}}

\lstdefinelanguage{LFortran}[]{Fortran}{
  morekeywords={abstract,import,type,class,extends,implements,pass,associate}
}

\hypersetup{
    colorlinks,
    linkcolor={blue!70!black},
    citecolor={blue!70!black},
    urlcolor={blue!70!black}
}

\frenchspacing

\begin{document}

\title{A trait system for the uniform expression of run-time
       and compile-time polymorphism in Fortran}

\author{Konstantinos Kifonidis, Ondrej Certik, Derick Carnazzola}

\maketitle

\abstract{Based on conclusions drawn from a survey of modern
  languages, a trait system for Fortran is developed and described
  that is fully backwards compatible with the present Fortran
  language, and allows for the uniform management of source code
  dependencies on both user-defined and language-intrinsic types, via
  run-time and compile-time polymorphism (i.e. ``object-oriented'' and
  ``generic'' programming). The feature set described here is small
  enough to facilitate a first prototype implementation in an open
  source compiler, like LFortran, but at the same time comprehensive
  enough to already endow Fortran with polymorphism capabilities that
  largely equal those of modern programming languages like Swift,
  Rust, or Go. The discussed new features are expected to transform
  the way Fortran applications and libraries will be written in the
  future. Decoupled software plugin architectures with enormously
  improved source code flexibility, reusability, maintainability, and
  reliability will become possible, without any need for run-time type
  inspections, and without any loss in computational performance.}

\chapter{Introduction}

Polymorphism was discovered in the 1960s by Kristen Nygaard and
Ole-Johan Dahl during their development of Simula~67, the world's
first object-oriented (OO) language \cite{Dahl_04}. Their work
introduced into programming what is nowadays known as ``virtual method
table'' (i.e. function-pointer) based run-time polymorphism, which is
both the first focus of this document, and the defining feature of all
OO languages. Several other forms of polymorphism are known today, the
most important of them being parametric polymorphism
\cite{Cardelli_Wegner_85} (also known as ``generics''), which is the
second focus of this document, and which has historically developed
disjointly from run-time polymorphism, since it makes use of
compile-time mechanisms.


\section{The purpose of polymorphism}

But what is the purpose of polymorphism in a programming language?
What is polymorphism actually good for? One of the more comprehensive
answers to this question was given by Robert C. Martin in numerous
books (e.g. \cite{Martin_17}), as well as in the following quotation
from his blog \cite{Martin_14}:

\begin{displayquote}
``There really is only one benefit to polymorphism; but it's a big
  one. It is the inversion of source code and run time
  dependencies. In most software systems when one function calls
  another, the runtime dependency and the source code dependency point
  in the same direction. The calling module depends on the called
  module. However, when polymorphism is injected between the two there
  is an inversion of the source code dependency. The calling module
  still depends on the called module at run time. However, the source
  code of the calling module does not depend upon the source code of
  the called module. Rather both modules depend upon a polymorphic
  interface. This inversion allows the called module to act like a
  plugin. Indeed, this is how all plugins work.''
\end{displayquote}

Notice, the absence of the words ``code reuse'' in these statements.
The purpose of polymorphism, according to Martin, is the ``inversion''
(i.e. replacement, or management) of source code dependencies by
(means of) particular abstractions, i.e. polymorphic interfaces (or
protocols/traits, as they are also known today). The possibility to
reuse code is then merely the logical consequence of such proper
dependency management.

\section{Source code dependencies in statically typed languages}

Which then are the source code dependencies that polymorphism helps us
manage? It has been customary to make the following distinction when
answering this question:
\begin{itemize}
\item
  Firstly, most larger programs that are written in statically typed
  languages (like Fortran) have dependencies on \emph{user-defined}
  procedures and data types. If the programmer employs encapsulation
  of both a program's procedures and its data, i.e. its state, both
  these dependencies can actually be viewed as dependencies on
  user-defined abstract data types. That is, types that contain both
  user-defined state, and implementation code which operates on that
  (hidden) state. These are the dependencies that Martin is concretely
  referring to in the above quotation, and it is these dependencies on
  (volatile) implementation (details) that are particularly
  troublesome, because they lead to rigid coupling between the various
  different \emph{parts} of an application. Their results are
  recompilation cascades, the non-reusability of higher-level modules,
  the impossibility to comprehend a large application incrementally,
  and fragility of such an application as a whole.
\item
  Secondly, every program, that is written in a statically typed
  language, also has dependencies on abstract data types that are
  provided by the language itself. Fortran's \code{integer},
  \code{real}, etc. intrinsic types are examples of language intrinsic
  abstract data types. While hard-wired dependencies on such intrinsic
  types may not couple different parts of a program (because the
  implementations of these types are supplied by the language), they
  nevertheless make a program's source code rigid with respect to the
  data that it can be used on.
\end{itemize}

The most widely used approaches to manage dependencies on language
intrinsic types have so far been through generics, while dependency
management of user-defined (abstract data) types has so far been the
task of OO programming and OO design patterns. Martin \cite{Martin_17}
has, for instance, defined object-orientation as follows:

\begin{displayquote}
  ``OO is the ability, through the use of polymorphism, to gain
  absolute control over every source code dependency in [a software]
  system. It allows the architect to create a plugin architecture, in
  which modules that contain high-level policies are independent of
  modules that contain low-level details. The low-level details are
  relegated to plugin modules that can be deployed and developed
  independently from the modules that contain high-level policies.''
\end{displayquote}

\section{Modern developments}

Notice how Martin's modern definition of object-orientation, that
emphasizes source code decoupling, is the antithesis to the usually
taught ``OO'' approaches of one class rigidly inheriting
implementation code from another. Notice also how his definition does
not require some specific type of polymorphism for the task of
dependency management, as long as (according to Martin's first
quotation) the mechanism is based on polymorphic interfaces.

Martin's statements on the purpose of both polymorphism and OO simply
reflect the two crucial developments that have taken place in these
fields over the last decades. Namely, the realizations that
\begin{itemize}
\item
  run-time polymorphism should be freed from the conflicting concept
  of implementation inheritance (to which it was originally bound
  given its Simula~67 heritage), and be formulated exclusively in
  terms of conformance to polymorphic interfaces, i.e. function
  signatures, or purely procedural abstractions, and that
\item
  compile-time polymorphism should be formulated in exactly the same
  way as well.
\end{itemize}

These two developments taken together have recently opened up the
possibility to treat polymorphism, and hence the dependency management
of both user-defined and language intrinsic types, uniformly in a
programming language. As a consequence, it has become possible to use
the potentially more efficient (but also less flexible) mechanism of
compile-time polymorphism also for a number of tasks that have
traditionally been reserved for run-time polymorphism (i.e. OO
programming), and to mix and match the two polymorphism types inside a
single application to better satisfy a user's needs for both
flexibility and efficiency.

\section{Historical background}

The road towards these realizations has been surprisingly long. Over
the last five decades, a huge body of OO programming experience first
had to demonstrate that the use of (both single and multiple)
implementation inheritance breaks encapsulation in OO languages, and
therefore results in extremely tightly coupled, rigid, fragile, and
non-reusable code. This led to an entire specialized literature on OO
design patterns \cite{Gamma_et_al_94,Martin_03,Holub_04}, that aimed
at avoiding or mitigating the effects of such rigidity, by replacing
the use of implementation inheritance with the means to formulate
run-time polymorphism that are discussed below. It also led to the
apprehension that implementation inheritance (but \emph{not} run-time
polymorphism) should be abandoned \cite{Weck_Szyperski}. In modern
languages, implementation inheritance is either kept solely for
backwards compatibility reasons (e.g. in the Swift and Carbon
languages), or it is foregone altogether (e.g. in Rust, and Go).

The first statically typed mainstream programming language that
offered a proper separation of run-time polymorphism from
implementation inheritance was Objective~C. It introduced
``protocols'' (i.e. polymorphic interfaces) in the year 1990
\cite{Cox_et_al_20}. Protocols in Objective C consist of pure function
signatures, that lack implementation code. Objective~C provided a
mechanism to implement multiple such protocols by a class, and to thus
make classes conform to protocols. This can be viewed as a restricted
form of multiple inheritance, namely inheritance of object
\emph{specification}, which is also known as \emph{subtyping}. Only a
few years later, in 1995, the Java language hugely popularized these
ideas using the terms ``interfaces'' and ``interface inheritance''
\cite{Cox_et_al_20}. Today, nearly all modern languages support
polymorphic interfaces/protocols, and the basic mechanism of multiple
interface inheritance that was introduced to express run-time
polymorphism in Objective~C, often in even improved, more flexible,
manifestations. The only negative exceptions in this respect being
modern Fortran, and C++, which both still stick to the obsolescent
Simula~67 paradigm.

A similar learning process, as that outlined for run-time
polymorphism, took place in the field of compile-time/parametric
polymorphism. Early attempts, notably templates in C++, to render
function arguments and class parameters polymorphic, did not impose
any constraints on such arguments and parameters, that could be
checked by C++ compilers. With the known results on compilation times
and cryptic compiler error messages
\cite{Haveraaen_et_al_19}. Surprisingly, Java, the language that truly
popularized polymorphic interfaces in OO programming, did not provide
an interface based mechanism to constrain its generics. Within the
pool of mainstream programming languages, this latter realization was
first made with the advent of Rust \cite{Matsakis_2014}.

Rust came with a trait (i.e. polymorphic interface) system with which
it is possible for the user to uniformly and transparently express
both generics (i.e. compile-time) and run-time polymorphism in the
same application, and to relatively easily switch between the two,
where possible. Rust's traits are an improved form of
protocols/interfaces in that the user can implement them for a type
without having these implementations be coupled to the type's actual
definition. Thus, existing types can be made to retroactively
implement new traits, and hence be used in new settings (with some
minor restrictions on user ownership of either the traits or the
types).

Rust's main idea was quickly absorbed by almost all other mainstream
modern languages, most notably Go, Swift, and Carbon, with the
difference that these latter languages tend to leave the choice
between static and dynamic procedure dispatch to the compiler, or
language implementation, rather than the programmer. C++ is in the
process of adopting generics constraints for its ``templates'' under
the term ``strong concepts'', but without implementing the greater
idea to uniformly express \emph{all} the polymorphism in the language
through them. An implementation of this latter idea must today be
viewed as a prerequisite in order to call a language design
``modern''. The purpose of this document is to describe additions to
Fortran, that aim to provide the Fortran language with such modern
capabilities.

\chapter{Case Study: Calculating the average value of a numeric array}

To illustrate the advanced features and capabilities of some of the
available modern programming languages with respect to polymorphism,
and hence dependency management, we will make use here of a case
study: the simple test case of calculating the average value of a set
of numbers stored inside a one-dimensional array. In the remainder of
this chapter we will first provide an account and some straightforward
monomorphic (i.e. coupled) functional implementation of this test
problem, followed by a functional implementation that makes use of
both run-time and compile-time polymorphism to manage source code
dependencies. In the survey of programming languages presented in
Chapter~\ref{sect:survey}, we will then recode this test problem in an
encapsulated fashion, to highlight how the source code dependencies in
this problem can be managed in different languages even in more
complex situations, that require OO techniques.

\section{Monomorphic functional implementation}
\label{sect:mono_functional}

We have chosen Go here as a language to illustrate the basic ideas.
Go is easily understood, even by beginners, and is therefore well
suited for this purpose (another good choice would have been the Swift
language). The code in the following Listing~\ref{lst:funcGo} should
be self explanatory for anyone who is even only remotely familiar with
the syntax of C family languages. So, we'll make only a few remarks
regarding syntax.
\begin{itemize}
\item
  While mostly following a C like syntax, variable declarations in Go
  are essentially imitating Pascal syntax, where a variable's name
  precedes the declaration of the type.
\item
  Go has two assignment operators. The usual \code{=} operator, as it
  is known from other languages, and the separate operator \code{:=}
  that is used for combined declaration and initialization of a
  variable.
\item
  Go has array slices that most closely resemble those of Python's
  Numpy (which exclude the upper bound of an array slice).
\end{itemize}

Our basic algorithm for calculating the average value of an array of
integer elements employs two different implementations for
averaging. The first makes use of a ``simple'' summation of all the
array's elements, in ascending order of their array index. While the
second sums in a ``pairwise'' manner, dividing the array in half to
carry out the summations recursively, and switching to the ``simple''
method once subdivision is no longer possible.

As a result, this code has three levels of hard-wired (i.e. rigid)
dependencies. Namely,
\begin{enumerate}
\item
  function \code{pairwise\_sum} depending on function
  \code{simple\_sum}'s implementation,
\item
  functions \code{simple\_average} and \code{pairwise\_average}
  depending on functions' \code{simple\_sum}, and \code{pairwise\_sum}
  implementation, respectively, and
\item
  the entire program depending rigidly on the \code{int32} data type in
  order to declare both the arrays that it is operating on, and
  the results of its summation and averaging operations.
\end{enumerate}
The first two items are dependencies on user-defined implementations,
while the third is a typical case of rigid dependency on a language
intrinsic type, which renders the present code incapable of being
applied to arrays of any other data type than \code{int32}s. Given that
we are dealing with three levels of dependencies, three levels of
polymorphism will accordingly be required to remove all these
dependencies.

\lstinputlisting[language=Go,style=boxed,label={lst:funcGo},caption={Monomorphic functional version of the array averaging example in Go.}]{Code/Go/coupled.go}

\section{Polymorphic functional implementation}
\label{sect:poly_functional}

Listing~\ref{lst:polyfuncGo} gives an implementation of our test
problem, that employs Go's generics and functional features in order
to eliminate the last two of the rigid dependencies that were listed
in Sect.~\ref{sect:mono_functional}. The code makes use of Go's
generics to admit arrays of both the \code{int32} and \code{float64}
types as arguments to all functions, and to express the return values
of the latter. It also makes use of the run-time polymorphism inherent
in Go's functional features, namely closures and variables of
higher-order functions, to replace the two previous versions of
function \code{average} (that depended on specific implementations),
by a single polymorphic version. Only the rigid dependency of function
\code{pairwise\_sum} on function \code{simple\_sum} has not been
removed, in order to keep the code more readable. In the OO code
versions, that will be presented in Chapter~\ref{sect:survey}, even
this dependency is eliminated.

A few remarks are in order for a better understanding of
Listing~\ref{lst:polyfuncGo}'s code:
\begin{itemize}
\item
  In Go, generic type parameters to a function, like the parameter
  \code{T} here, are provided in a separate parameter list, that is
  enclosed in brackets [ ].
\item
  Generic type parameters have a constraint that follows their
  declared name. Go exclusively uses interfaces as such constraints
  (see the interface \code{INumeric} in the present example).
\item
  Interfaces consist of either function signatures, or \emph{type
  sets}, like ``\code{int32 | float64}'' in the present example. The
  latter signify a set of function signatures, too, namely the
  signatures of the intersecting set of all the operations/functions
  for which the listed types provide implementations.
\item
  The code makes use of type conversions to the generic type \code{T},
  where required. For instance, \code{T(0)} converts the constant
  \code{0} to the corresponding zero constant of type \code{T}.
\item
  The code instantiates closures and stores these by value in two
  variables named \code{avi} and \code{avf} for later use (Fortran
  and C programmers should note that \code{avi} and \code{avf} are
  \emph{not} function pointers!).
\end{itemize}


\lstinputlisting[language=Go,style=boxed,label={lst:polyfuncGo},caption={Polymorphic functional version of the array averaging example in Go.}]{Code/Go/functional.go}

The motivation to code the example as in Listing~\ref{lst:polyfuncGo}
is that once the two closures, \code{avi}, and \code{avf}, are properly
instantiated (by means of the \code{switch} statement), they may be
passed from the main program to any other client code that may need to
make use of the particular averaging algorithm that was selected by
the user. This latter client code would \emph{not} have to be littered
with \code{switch} statements itself, and it would \emph{not} have to
depend on any specific implementations. It would merely depend on the
closures' interfaces. The same holds for the OO code versions that are
discussed in the next chapter, with objects replacing the closures
(both being merely slightly different realizations of the same idea).

\chapter{Survey of modern languages}
\label{sect:survey}

In the present chapter we give implementations, in various modern
languages, of encapsulated (i.e. OO) code versions of the test
problem. As in the functional code version presented in
Sect.~\ref{sect:poly_functional}, we employ run-time polymorphism to
manage the dependencies on user-defined implementations (in this case
abstract data types), and generics in order to manage the dependencies
on language intrinsic types. This serves to illustrate how both
run-time and compile-time polymorphism can be typically used for
dependency management in an OO setting in these modern languages. The
survey also aims to highlight the many commonalities but also some
of the minor differences in the approaches to polymorphism that were
taken in these different languages. As a final disclaimer, we do not
advocate to code problems in an OO manner that can be easily coded in
these languages in a functional way (as it is the case for this
problem). However, in more complex cases, where many more nested
functions would need to be used, and where state would have to be
hidden, the OO programming style would be the more appropriate
one. Hence our test problem will stand in, in this chapter, for
emulating also such a more complex problem, that would benefit from an
encapsulated coding style.


\section{Go}

Go has supported run-time polymorphism through (polymorphic)
``interfaces'' (and hence modern-day OO programming) since its
inception. In Go, encapsulation is done by storing state in a
``\code{struct}'' and by binding procedures, that need to use that
state, to this same \code{struct}. Thus creating a user-defined
abstract data type (or ADT) with methods. Go allows the programmer to
implement multiple polymorphic interfaces for such a type (i.e. to use
multiple interface inheritance), even though it offers no explicit
language statement for this purpose.

Instead, a user-defined type is implicitly assumed to implement an
interface whenever it provides implementations of all the interface's
function signatures. This way of implementing interfaces requires only
an object reference of the type to be passed to its methods (by means
of a separate parameter list, in front of a method's actual name). It
is otherwise decoupled from the type's (i.e. the ADT's \code{struct})
definition. Limitations in Go are that language intrinsic types cannot
have methods, and that methods and interfaces cannot be directly
implemented for user-defined types whose definitions are located in
other packages. That is, the programmer has to write wrappers in the
latter case. Go, finally, makes it explicit in its type definition
syntax that interfaces (like \code{struct}s) are types in their own
right, and that hence polymorphic variables (i.e. objects) can be
declared in terms of them.

Since version 1.18, Go also supports compile-time polymorphism through
generics. Go's generics make use of ``strong concepts'', since they
are bounded by constraints that are expressed through
interfaces. Hence, the Go compiler will fully type-check generic code.
In Go, structures, interfaces, and functions, but not methods, can all
be given their own generic type parameters.

\subsection{Encapsulated version coded in Go}

Listing~\ref{lst:OOGo} gives an encapsulated version of the test
problem coded in Go. The two different implementations of the
\code{sum} function have been encapsulated in two different ADTs named
\code{SimpleSum} and \code{PairwiseSum}, whereas a third ADT named
\code{Averager} encapsulates the functionality that is required to
perform the actual averaging. The latter two ADTs contain the
lower-level objects ``\code{other}'' and ``\code{drv}'' of
\code{ISum[T]} type as components, to which they delegate calls to
these objects' \code{sum} methods. Notice how the use of the
polymorphic interface \code{ISum[T]} in the declarations of
\code{other} and \code{drv} enables either \code{SimpleSum} or
\code{PairwiseSum} objects to be plugged into their higher-level
clients.

A second interface, named \code{IAverager}, is used to enable
polymorphism for different averaging algorithms. Finally, there's a
third interface, \code{INumeric}, that serves exactly the same purpose
as in the functional polymorphic version given in
Sect.~\ref{sect:poly_functional}, namely to make all function
arguments and return values polymorphic, by admitting as input and
output parameters both the \code{int32} and \code{float64} intrinsic
types.

Hence, three polymorphic interfaces were required in this code, in
order to eliminate the three levels of rigid dependencies that were
listed in Sect.~\ref{sect:mono_functional}. Notice also that,
exempting \code{INumeric}, all the interfaces and all the user-defined
ADTs need to take in generic type parameters in this example. This is
required in order to enable all the \code{sum} and \code{average}
methods to use generic type parameters in Go.

\lstinputlisting[language=Go,style=boxed,label={lst:OOGo},caption={Encapsulated Go version of the array averaging example.}]{Code/Go/mixed.go}

The main program makes use of Go's built-in structure constructors,
and constructor chaining, in order to instantiate objects of the
required ADTs. In particular, it instantiates run-time polymorphic
``\code{Averager}'' objects (depending on whether simple or pairwise
sum averaging is to take place), and it does so for both the
\code{int32} and \code{float64} types separately, in order to then use
these objects on \code{int32} and \code{float64} data, respectively. The
fact that \code{two} such objects are required (one for each language
intrinsic data type) is connected to the fact that in order to obtain
generic methods in Go, one has to parameterize interfaces by generic
parameters, and instantiate them with different actual data types, as
in \code{func main}'s first two code lines. A single
(i.e. unparameterized) \code{IAverager} interface therefore doesn't
suffice, which is unfortunate from the user's perspective, as some
code duplication in client code cannot be avoided in this way.

\section{Rust}

Like Go, Rust supports both run-time and compile-time polymorphism
through polymorphic interfaces, which Rust calls ``traits''. In
contrast to Go, Rust has its programmers implement traits in an
explicit manner, by using explicit ``\code{impl}'' code blocks to
provide a trait's method implementations. These same \code{impl}
blocks also serve to bind methods to a type that aren't a part of some
trait, like e.g. user-defined constructors for \code{struct}s (see the
functions named ``\code{new}'' in the following code
Listing~\ref{lst:OORust}).

Differing from Go, Rust allows the programmer to implement traits for
both user-defined \emph{and} language intrinsic types, and to do so
for types that are located in external libraries (called ``crates'' in
Rust), as long as the traits themselves are defined in the
programmer's own crate. The reverse, namely implementing an external
trait for a user-owned type, is also possible. Only the (edge) case of
implementing an external trait for an external type is not allowed
(this is called the ``orphan rule'' \cite{Klabnik_Nichols}). The
latter case requires the use of wrappers.

Comparable to Go, Rust's generics model allows for the generic
parameterization of functions, traits, and user-defined types like
\code{struct}s. Rust does not explicitly forbid generic
methods. However, if one defines such a method within a trait, then
this will make the trait unusable for the declaration of any ``trait
objects'' \cite{Lyon}, i.e. for the employment of run-time
polymorphism. Thus, the Rust programmer will in general (need to)
parameterize traits and \code{struct}s rather than any methods
themselves. Rust generics are fully type-checked at compilation time,
i.e. Rust supports ``strong concepts''.

\subsection{Encapsulated version coded in Rust}

The encapsulated Rust version of our test problem that is given in the
following Listing~\ref{lst:OORust} is in its outline quite similar to
the corresponding Go version. There are, however, a few minor
differences, that are listed in the following notes.

\begin{itemize}
\item
  Rust uses angled brackets to indicate generic parameter lists.
\item
  Generics constraints in Rust are typically enforced by specifying
  the required traits in \code{impl} blocks using \code{where}
  statements.
\item
  Use of a ``\code{Num}'' trait from the external ``\code{num}'' crate was
  necessary, in order to enable numeric operations on generic types,
  which leads to dependency on external library code.
\item
  At times, use of the ``\code{Copy}'' trait also had to be made, to
  work around Rust's default move semantics.
\item
  In order to help make all of the source code dependencies explicit,
  our Rust version employs modules, and \code{use} statements to import
  the required functionality.
\item
  Despite reliance on external dependencies, conversion to generic
  types wasn't possible. This led to the necessity to move the type
  conversion from method \code{average} to its calls in the main
  program. We also had to import a \code{zero} generic function from
  the external \code{num} crate, in order to initialize the
  variable \code{s} that is returned by the \code{sum} method of the
  \code{SimpleSum} ADT.
\item
  Rust's default structure constructors suffer from the same flaw as
  Fortran's. That is, they are unable to initialize from an external
  scope, structure components that are declared being private to their
  module. As in Fortran, use of user-defined constructors must be made
  instead (see the functions named \code{new} that are defined in
  separate \code{impl} blocks for the ADTs \code{PairwiseSum} and
  \code{Averager}).
\item
  To declare run-time polymorphic variables one has to put so-called
  ``trait objects'' into ``Boxes'', i.e. to declare smart pointers of
  them, for dynamic instantiation and memory allocation (this is the
  Rust equivalent to using \code{allocatable} polymorphic objects in
  Fortran).
\end{itemize}

\lstinputlisting[language=Rust,style=boxed,label={lst:OORust},caption={Encapsulated Rust version of the array averaging example.}]{Code/Rust/mixed_poly/src/main.rs}

The main program in the Rust version is somewhat longer then in the
corresponding Go version because of the need to import dependencies
from modules (as it would be necessary in realistic situations). Its
logic is also somewhat convoluted compared to the Go version, because
Rust doesn't allow the programmer to declare variables that aren't
initialized upon declaration, and because of the aforementioned
necessity to move the required type conversions out of method
\code{average}, and into the calls of this method. Otherwise the codes
are pretty much identical.


\section{Swift}

Being a successor language to Objective~C, Swift differs slightly from
the languages considered so far in that it opted to retain
implementation inheritance for backwards compatibility to Objective~C,
whereas both Go and Rust do not support implementation inheritance
\emph{by design}. Swift therefore supports ``classical'' classes, but
it also allows one to bind methods to structures (which, in contrast
to classes, are value types in Swift).

Like Go and Rust, Swift (furthermore) supports a trait system in order
to implement both run-time and compile-time polymorphism through
polymorphic interfaces, that are called ``protocols'' in Swift. If the
Swift programmer chooses to ignore implementation inheritance and
classes, he can therefore very much program with structures and
protocols in Swift as he would with structures and interfaces/traits
in Go and Rust, respectively.

Given Swift's backwards compatible design, implementation of a
protocol (i.e. interface inheritance) is usually done as in
``classical'' OO languages, i.e. within a structure's or a class's
definition. The ``\code{:}'' operator followed by one or more
interface names must be supplied for this purpose after the
structure's or class's own name. However, a very powerful facility for
types to implement protocols retroactively is also provided, through
so-called ``\code{extension}s'', that work even if the types' source
code is inaccessible (because one is, e.g., working with a library in
binary form). This same facility also allows the implementation of
protocols for language-intrinsic types. For instance, the following
little program prints out ``\code{I am 4.9}'':
\lstinputlisting[language=Swift,style=boxed]{Code/Swift/extension.swift}

Swift generics support ``strong concepts'', and are thus fully
type-checked at compile time, and their capabilities are on par with
those of Go and Rust. In one aspect they are even superior, namely in
that Swift allows for parameterized \emph{methods}, instead of
parameterized protocols. This has some interesting, positive
implications for the Swift programmer, that will be discussed in
detail below.

\subsection{Encapsulated version coded in Swift}

Listing~\ref{lst:OOSwift} gives an example of how the encapsulated
version of the array averaging test problem can be programmed in
Swift. See the following remarks in order to understand this code:

\begin{itemize}
  \item
    Swift uses angled brackets \code{<>} to indicate generic parameter
    lists.
  \item
    Type constraints are formulated by supplying a protocol name after a
    type parameter (separated by a colon).
  \item
    Swift does not supply an equivalent to Go's \code{int32 | float64}
    syntax. Hence the user must use a \code{Numeric} protocol defined
    by the standard library, as a constraint for numeric types. Which
    leads to reliance on library code.
  \item
    Unfortunately, Swift's \code{Numeric} protocol does \emph{not} support
    the division operation! Hence the division that would have been required
    in function \code{average} of the \code{Averager} ADT had to be moved
    out to the calling code of the main program.
  \item
    The Swift version makes use of language built-in, default, structure
    constructors (called ``initializers'').
  \item
    Array slices are not arrays themselves. So an explicit conversion
    using an \code{Array()} constructor is required in such cases.
  \item
    By default, function and method calls in Swift make use of keyword
    arguments.
  \item
    The syntax for type conversion into a generic type \code{T} is
    somewhat peculiar. E.g. Go's \code{T(0)} is written as
    \code{T(exactly:0)!} in Swift (making use of the mandatory keyword
    ``\code{exactly}'' in the function responsible for the type
    conversion).
\end{itemize}

\lstinputlisting[language=Swift,style=boxed,label={lst:OOSwift},caption={Encapsulated Swift version of the array averaging example.}]{Code/Swift/mixed.swift}

Even a casual glance at the Swift version will show that the Swift
code is the easiest to read and understand among all the encapsulated
implementations. This is largely the result of Swift supporting
generic methods, and hence not requiring the programmer to
parameterize and instantiate any generic interfaces/protocols, in
contrast to both Go and Rust. The consequences are
\begin{itemize}
\item
that method genericity for an ADT's objects can be expressed using
only a single, as opposed to multiple protocols,
\item
that merely a \emph{single} object instance of that same protocol is
required, in order to be able to operate on many different language
intrinsic data types, and
\item
that this also largely \emph{obviates the need for manual instantiations
of generics in Swift} (because generic functions/methods are easier to
instantiate automatically by the compiler, as it can always infer the
required types by checking the actual arguments that are passed to a
function/method)!
\end{itemize}

As an example, consider the \code{IAverager} protocol in the above
Swift code. There's only a single (i.e. unparameterized) version of
this protocol. Consequently, there's only a need in the main program
to declare a single object variable, \code{av}, of that protocol (that
enables \code{av} to be polymorphically assigned different
\code{struct}s that implement \code{IAverager}). Because it contains
an ``\code{average}'' method that is generic, this \emph{single}
object can then be straightforwardly used on data of \emph{both} the
\code{Int32} and \code{Float64} types!

This vastly simplifies client code that needs to make use of objects
such as \code{av}, especially if such client code needs to work on
\emph{many} more types than just \code{Int32} and
\code{Float64}. Contrast this with Go's and Rust's model, where manual
instantiation of a different version of \code{IAverager} is required
for \emph{every} different generic type parameter that the user wishes
to employ. Notice also, how there's \emph{not a single manual
instantiation} of generics code required in the Swift example! We
consider these significant advantages of the generics approach that is
taken in Swift vs. that of Go and Rust.


\section{Conclusions}

The use of run-time polymorphism by means of interfaces is rather
similar in all the languages considered here. The most significant
differences (that were not concretely explored here) appear to be that
Go has stricter limitations on retroactively implementing interfaces
for existing types than the other languages. Whereas Rust (with some
minor restrictions), and Swift allow the implementation of an interface
by some type to be accomplished independently from the type's
definition site. Rust and Swift thereby overcome Haveraaen et al.'s
critique \cite{Haveraaen_et_al_19} of Java regarding this point. In
fact, it is \emph{interface inheritance} which makes the uniform
polymorphic treatment of both intrinsic and user-defined types
possible in the first place in Rust and Swift, that Haveraaen et
al. seem to also (rightly) demand. In the following we will make some
final comments on the differences in all these languages' generics
features.

\subsection{Go}

Go's basic model to implement generics allows structures, interfaces,
and ordinary functions, but not methods, to be given their own generic
type parameters. The lack of true generic methods makes some
duplication of instantiation code in clients
unavoidable. Nevertheless, generic Go code is quite easy to read and
to understand. What sets Go apart from the other languages is its
built-in, easy to use support for conversion to generic types, and
especially its brilliant new notion to interpret interfaces as type
sets, along with its syntax to support this notion. This enables the
Go programmer to easily tailor constraints on generic types to his
specific needs, which is what makes the use of generics in Go
pleasant. We consider these latter particular features of Go as ``must
haves'' for Fortran.

\subsection{Rust}

Rust's basic model for generics is similar to Go's in that it allows
for parameterization of structures, interfaces, and ordinary
functions. Hence, what has been said above for Go in this respect
holds also for Rust. Rust has, unfortunately, some quirks which render
its use for the management of all types of dependencies through
polymorphism somewhat sub-optimal when compared to the other languages
considered here. The language is unpleasant to use, because of its
``borrow checker'', its \emph{excessive} obsession with type safety,
its employment of move semantics by default, and its overall C++-like
philosophy to copiously rely on external dependencies, even for the
most basic tasks, like initializing a generic type. The Rust version
of our test case is therefore marred by some dependencies on external
libraries, which is somewhat contrarian to the purpose of programming
in a polymorphic fashion, namely to avoid rigid dependencies. But even
with the functionality provided by such external dependencies, Rust
doesn't allow type conversion to generic types within generic
routines. A necessary capability for numerical work that is, for
instance, built into Go. The points we like most about the language
are its idea to decouple trait implementations from a \code{struct}'s
definition through explicit \code{impl} blocks, and the complete
control over the use of dynamic vs. static dispatch that Rust affords
the programmer. These are particular features of Rust that, in our
opinion, Fortran should borrow in some form.

\subsection{Swift}

Swift's basic model of implementing generics by allowing parameterized
structures, functions, and methods (but not parameterized interfaces)
is both the easiest to read, and the easiest to use from a
programmer's perspective. Swift's generics design allows the Swift
compiler to instantiate generics largely automatically, through
inspection of the argument types that are passed to functions,
methods, and (structure or class) constructors. In contrast to the
other languages, in Swift, the user basically never has to bother with
instantiating any generics.

If the Swift programmer knows how to write generic functions, his
knowledge automatically translates into coding generic methods, since
generic functions can be transformed into generic methods without
requiring any changes to their function signatures. This property is
helpful for the refactoring of non-OO codes into corresponding OO
versions.

We hence consider Swift's generics to be the most attractive model to
base Fortran's basic generic capabilities on, provided that it can be
implemented sufficiently easily. The fact that Swift is a language
that does not put emphasis on numerics, and whose present standard
library therefore does not provide a truly useful \code{Numeric}
protocol (that supports all the usual numeric operations), is of
absolutely no consequence for adopting Swift's basic generics design
for Fortran.

Fortran will necessarily do a better job in this respect, both by
borrowing Go's idea of interpreting type sets as interfaces, so that
the user can easily implement his own type constraints. But also by
making accessible to the user a set of language-built in interfaces
that are truly useful for numeric operations, and are implemented by
Fortran's intrinsic types.


\chapter{Fortran additions I: Subtyping}

The present and the next chapter describe additions to Fortran that we
consider essential in order to enable dependency management through
polymorphism at a level of functionality that is on par with modern
languages like Swift, Rust, or Go.

\section{Named abstract interfaces (traits)}

The most important of the following additions is the capability to
define named abstract interfaces, or traits (i.e. named collections of
procedure signatures), and to declare instance variables of
them. Named abstract interfaces are the crucial feature that is
required in order to uniformly and properly express both run-time and
compile-time polymorphism (i.e. generics) in the language, and to
thereby enable a uniform management of dependencies on both
user-defined \emph{and} language-intrinsic types.

\subsection{Abstract interface definitions}
\label{sect:interface_defs}

Fortran already allows the programmer to define unnamed abstract
interfaces, but in order to use these as types, named versions of them
are required, as in the following example, that defines two such named
interfaces, \code{IAddition} and \code{ISubtraction}, that are
intended as abstract blueprints for actual implementations of two
type-bound procedures, named \code{add} and \code{sub}:
\begin{lstlisting}[language=LFortran,style=boxed]
   abstract interface :: IAddition
      subroutine add(self,b)
         import; implicit none
         class(IAddition), intent(inout) :: self
         real,             intent(in)    :: b
      end subroutine add
   end interface

   abstract interface :: ISubtraction
      subroutine sub(self,b)
         import; implicit none
         class(ISubtraction), intent(inout) :: self
         real,                intent(in)    :: b
      end subroutine sub
   end interface
\end{lstlisting}
Since this is a simple addition to Fortran, that merely aims to
further extend the use cases of abstract interfaces in the language
(which presently serve as bounds on the signatures of procedure
pointers, and \code{deferred}, i.e. abstract, methods), it is fully
backwards compatible.

\subsection{Variable declarations}

Named abstract interfaces/traits are types in their own right. Their
purpose is to allow the programmer to declare variables in terms of
them. Either directly, i.e. as objects of such interfaces in run-time
polymorphism, or as constraints on generic type parameters in
compile-time polymorphism (see Sect.~\ref{sect:generic_parameters} for
examples of the latter).

\subsubsection{\code{Class} declaration specifier: enhancements}

In order to use abstract interfaces to support run-time
polymorphic objects through subtyping, Fortran's \code{class}
specifier for variable declarations needs to be enhanced to accept
named abstract interfaces, like in the following two examples:
\begin{lstlisting}[language=LFortran,style=boxed]
  class(IAddition), allocatable :: adder
  class(IAddition), pointer     :: adderptr
\end{lstlisting}
The semantics here are that whenever a named abstract interface
appears within the \code{class} specifier of an object's declaration,
then all the \code{public} methods of that object, whose signatures
are prescribed by an adopted interface like \code{IAddition} here,
will make use of late binding. That is, their calls will be resolved
by the run-time system of the language (e.g. through a virtual method
table). In accordance with how objects that make use of run-time
polymorphism through subclassing (i.e. implementation inheritance) are
declared in the present Fortran standard, also ``trait objects'' (like
\code{adder} or \code{adderptr} in the example above) must either be
declared using the \code{allocatable}, or the \code{pointer}
attribute, or they must be arguments to a procedure (as in the example
of Sect.~\ref{sect:interface_defs}). The proposed additions are
therefore backwards compatible with the functionality that is already
available in the present language.

\subsubsection{\code{Class} declaration specifier: constraints}
\label{sect:class_specifier_constraints}

Since the \code{class} declaration specifier is undefined for
intrinsic types, and since the aforementioned semantics of this
specifier imply late binding of methods (which is incompatible with
intrinsic types), a compiler will need to ensure that the \code{class}
declaration specifier is not used in conjunction with interfaces that
are intended for use as generics constraints and therefore admit
language intrinsic types, i.e. interfaces that are formulated in terms
of type sets, as they are discussed in
Sect.~\ref{sect:type_sets}. These include the empty interface,
\code{IAnyType}.

\subsection{Extends specifier for abstract interfaces}

Abstract interface definitions must allow the programmer to declare
new abstract interfaces that inherit procedure signatures from
\emph{multiple} simpler interfaces (multiple interface
inheritance). In the following example, the interface
\code{IBasicMath} inherits the procedure signatures contained in both
the interfaces \code{IAddition}, and \code{ISubtraction}, making
\code{IBasicMath} at the same time a \emph{subtype} of both these
simpler interfaces. That is, objects that adopt or implement the
\code{IBasicMath} interface (i.e. conform to it), can also be used in
settings that require conformance to either the \code{IAddition}, or
\code{ISubtraction} interfaces.
\begin{lstlisting}[language=LFortran,style=boxed]
   abstract interface, extends(IAddition,ISubtraction) :: IBasicMath
   end interface
\end{lstlisting}

\subsection{Short-hand notation for combination of abstract interfaces}

There are often cases where the combined functionality of two (or more)
interfaces is required but where one would not like to go through the
labor to explicitly set up a separate derived interface, like
\code{IBasicMath} above. This can be useful in variable
declarations. In such cases, it should be possible to specify the
following
\begin{lstlisting}[language=LFortran,style=boxed]
   class(IAddition + ISubtraction), allocatable :: addsub
\end{lstlisting}
instead of having to explicitly derive \code{IBasicMath} from
\code{IAddition} and \code{ISubtraction}, as above, and then use it as
follows:
\begin{lstlisting}[language=LFortran,style=boxed]
   class(IBasicMath), allocatable :: addsub
\end{lstlisting}


\section{Multiple interface inheritance for types}

The language must make it possible not only for named abstract
interfaces to conform to other named abstract interfaces, but also for
\emph{other types} to do the same, preferably regardless of whether
such types are user-defined or intrinsic to the language.


\subsection{\code{Implements} specifier for derived type definitions}

User-defined (i.e. derived) types can be made to conform to an
interface by introducing an \code{implements} specifier for derived
type definitions. In the following example the derived type
\code{BasicMath} implements (i.e. conforms to, or adopts) the
interface \code{IBasicMath} that was defined above
\begin{lstlisting}[language=LFortran,style=boxed]
module basic
   ...

   type, implements(IBasicMath) :: BasicMath
      private
      real :: a
   contains
      procedure, public, pass(self) :: add
      procedure, public, pass(self) :: sub
   end type BasicMath

contains

   subroutine add(self,b)
      class(BasicMath), intent(inout) :: self
      real,             intent(in)    :: b
      self%a = self%a + b
   end subroutine add

   subroutine sub(self,b)
      class(BasicMath), intent(inout) :: self
      real,             intent(in)    :: b
      self%a = self%a - b
   end subroutine sub
      
end module basic      
\end{lstlisting}
by providing implementations of all the method signatures that are
contained in that interface. Notice, how \code{BasicMath}, by virtue
of being an implementer of \code{IBasicMath}, is now also an
implementer of \code{IAddition}, and \code{ISubtraction}. Hence,
\code{BasicMath} can be used in client code that requires conformance
to either one of the interfaces \code{IBasicMath}, \code{IAddition},
or \code{ISubtraction}.

It is crucial, for flexibility, that the above interface inheritance
mechanism allow for a type to implement \emph{multiple} different
interfaces. For instance, if one wouldn't have defined the interface
\code{IBasicMath} from above, and would nevertheless need to use
objects of type \code{BasicMath} in settings that require conformance
to either the \code{IAddition}, or \code{ISubtraction} interfaces,
then the language must allow one to define type \code{BasicMath} as
follows (skipping, for brevity, the implementation of the actual
methods, that would be done exactly as in the previous example):
\begin{lstlisting}[language=LFortran,style=boxed]
   ...
   type, implements(IAddition,ISubtraction) :: BasicMath
   contains
      procedure, public, pass(self) :: add
      procedure, public, pass(self) :: sub
   end type BasicMath
   ...
\end{lstlisting}

In case the ``implementing'' type is \code{abstract}, it is allowed to
provide an only partial implementation of the interface(s) that it
adopts. However, any non-abstract type that is derived from this
abstract type through implementation inheritance (subclassing) must
then provide a full implementation.


\subsection{\code{Implements} statement}

In order to avoid having to wrap existing types into wrappers, when
unforeseen new use cases result, it should be possible (as in Swift or
Rust) to make any type implement new interfaces retroactively,
regardless of where its original type definition is located. It would
be desirable to provide such a capability for both user-defined and
language intrinsic types, which would mean that methods would have to
be allowed also for \emph{intrinsic} types. The \code{implements}
statement that is described below is aimed at ultimately accomplishing
these capabilities, but for the purpose of a first prototype
implementation it has been restricted here to provide such
functionality for user-defined types only, i.e. its use on
intrinsic types is presently prohibited.

Notice, that the \code{implements} statement has \emph{no} relation to
subclassing, i.e. one derived type being extended into another through
(rigid) implementation inheritance. Rather, this is a feature that
adds new capabilities to a single, \emph{given} type. The feature is
modeled after the ``\code{extension}'' feature of Swift, where it is
used to enable retroactive implementation of new methods, additional
constructors, and especially new interfaces for types, in order to
dynamically change an \emph{interface inheritance hierarchy}, and
achieve utmost code flexibility.

Swift's \code{extension} blocks fulfill essentially the same purpose
as Rust's \code{impl} blocks in this respect. They have been
simplified here (for a first implementation), and adjusted to
Fortran's syntax that binds methods to types through declaration
blocks, rather than by including the actual implementation bodies
themselves into such a block (the implementations need to be supplied
as module procedures, as is usual in Fortran). The syntax of the
feature is largely symmetric to that of the ``\code{contains}''
section of derived type definitions. Most of the options that are
allowed for type-bound procedure declarations in derived type
definitions, are therefore also allowed for such declarations within
an \code{implements} statement.

\subsubsection{Retroactively adding methods to a type}

Suppose that we would like to add, from within a different module, two
more methods to the \code{BasicMath} type from above, in order to give
this type some additional functionality. This can be done using an
\code{implements} \emph{statement} as follows:
\begin{lstlisting}[language=LFortran,style=boxed]
module enhanced

   use basic, only: BasicMath

   implicit none
   
   implements :: BasicMath
      procedure, public, pass(self) :: mul
      procedure, public, pass(self) :: div
   end implements BasicMath

contains
   ...
end module enhanced
\end{lstlisting}
where the actual implementations of the \code{mul} and \code{div}
procedures would be given after the module's \code{contains} statement
as usual.

\subsubsection{Retroactively implementing interfaces by a type}

Assume now that the purpose of our addition of the previous two
methods was to actually make \code{BasicMath} conform to settings
where implementations of multiplication with, or division by, a
\code{real} are needed, and where the required functionality is
described by two abstract interfaces called \code{IMultiplication} and
\code{IDivision}. So far, we have added the code of the required
methods, but we haven't made \code{BasicMath} pluggable into code that
is written in terms of either one of these latter interfaces. To fix
this, we simply state that the \code{BasicMath} type already has all
of the required functionality, by acknowledging this using an
\code{implements} statement for these two interfaces as follows:
\begin{lstlisting}[language=LFortran,style=boxed]
   ...
   use enhanced, only: BasicMath
  
   implements (IMultiplication,IDivision) :: BasicMath
   end implements BasicMath

\end{lstlisting}

We could have also skipped the latter two code examples, to instead
adopt the interfaces and provide the method implementations
simultaneously, splitting the \code{implements} statements into two, to
conform to one interface at a time, like so:
\begin{lstlisting}[language=LFortran,style=boxed]
   ...
   use basic, only: BasicMath
  
   implements IMultiplication :: BasicMath
      procedure, public, pass(self) :: mul
   end implements BasicMath

   implements IDivision :: BasicMath
      procedure, public, pass(self) :: div
   end implements BasicMath
   ...
\end{lstlisting}
The result would have been the same. Such splitting of
\code{implements} statements can be useful to improve readability, if
different interfaces contain multiple procedure signatures, that would
all have to be implemented. These two statements (together with the
required implementations), could then even be distributed among
different modules and files. Notice, also, how parentheses around
interface lists in \code{implements} statements are optional, but not
required.

\section{Interoperability with subclassing}

The present multiple interface inheritance (i.e. subtyping) features are
interoperable with the single implementation inheritance (i.e. subclassing)
that is already present in the language. That is, code examples like
the following are possible:
\begin{lstlisting}[language=LFortran,style=boxed]
   type :: Parent
   contains  
      procedure :: method1
      procedure :: method2
   end type Parent

   type, extends(Parent), implements(IChild) :: Child
   contains  
      procedure :: method3
      procedure :: method4
   end type Child
   ...
\end{lstlisting}
Here, a \code{Child} type is defined, that inherits two methods
(\code{method1} and \code{method2}) from a \code{Parent} type, and
adds two further methods of its own (\code{method3} and
\code{method4}), in order to conform to an interface, \code{IChild},
that consists of the signatures of all four of these methods. In such
use cases, the \code{extends} specifier shall always precede the
\code{implements} specifier.

Thus, the features described here are backwards compatible with the OO
model that is used in the present language. Moreover, since the new
\code{implements} specifier allows for inheritance of \emph{multiple}
interfaces (see above), this also fixes present Fortran's single
inheritance limitations without introducing the potential ambiguities
that multiple inheritance of implementation would cause (which are
also known as ``The Diamond Problem'').


\chapter{Fortran additions II: Generics}

The new subtyping features that were discussed in the previous chapter
are required in order to uniformly express and support both run-time
and compile-time polymorphism in Fortran. We will now proceed with
discussing some enhancements that are required in order to further
support compile-time polymorphism, i.e. generics.

\section{Enhancements to Interfaces}

\subsection{Generic procedure signatures} 
\label{sect:generic_interfaces}

Abstract interfaces should be allowed to contain signatures of generic
procedures, as in Swift. The approach taken in Go and Rust to
parameterize abstract interfaces themselves, appears not as attractive
from a user's perspective. The following code shows, as an example, an
abstract interface named \code{ISum} that contains the signature
intended for a generic type-bound procedure, named \code{sum}:
\begin{lstlisting}[language=LFortran,style=boxed]
   abstract interface :: ISum
      function sum{INumeric :: T}(self,x) result(s)
         import; implicit none
         class(ISum), intent(in) :: self
         type(T),     intent(in) :: x(:)
         type(T)                 :: s
      end function sum
   end interface
\end{lstlisting}

The example illustrates the use of a generic type parameter, that is
simply called \code{T} here, in terms of which the regular function
parameters are declared. A significant difference of generic type
parameters, as compared to regular function parameters, is that the
former will be substituted by an actual type argument at compile time,
in a process called instantiation. A similarity is that, in the same
way that regular function parameters need to be constrained by a
provided type, type parameters need to be constrained by a provided
meta-type. This (meta-type) constraint must be an abstract interface
name (like \code{INumeric} in the present example), that precedes the
actual type parameter. The proposed Fortran generics thus support
``strong concepts'', and can be fully type-checked by the
compiler. Both, the type parameter and its constraint, are part of a
generic type parameter list that is enclosed in curly braces, and
follows immediately behind the procedure's name. Notice that the
syntax used above, that deviates slightly from how Fortran's regular
function arguments are declared, appears justified, as it reflects
that, despite some similarities, in type parameters one is dealing
with different entities.

\subsection{Interfaces based on type sets}
\label{sect:type_sets}

In order to make the interface based generics facility easy to use for
the programmer, it must be possible, as in the Go language, to define
generics constraints by means of abstract interfaces that consist of
type sets.

\subsubsection{Unions of types}

The following example shows the simplest form of such a type set, by
defining an interface \code{INumeric}, for use as a generics
constraint in the example of Sect.~\ref{sect:generic_interfaces}, in
order to admit for the type parameter \code{T}, that was given there,
only the (32 bits wide) standard \code{integer} type:
\begin{lstlisting}[language=LFortran,style=boxed]
   abstract interface :: INumeric
      integer
   end interface
\end{lstlisting}

The above example is actually a special case of specifying entire
\emph{unions} of member types as a type set. A type set consisting of
such a union of types is demonstrated in the following example
\begin{lstlisting}[language=LFortran,style=boxed]
   abstract interface :: INumeric
      integer | real(real64)
   end interface
\end{lstlisting}
that redefines interface \code{INumeric} such as to admit either the
standard \code{integer} or the \code{real(real64)} type as a generics
constraint.

The semantics of such a type set construct is that it implicitly
defines a \emph{set of function signatures}, namely the signatures of
the intersecting (common) set of all the operations/intrinsic
functions that work with all the member types of the type set. This can
also be restated, by saying that a type \code{T} \emph{implements} an
interface consisting of such a type set, if (and only if) it is a
member of this set. The \code{complex} type, for instance, is not a
member of interface \code{INumeric}'s type set (as it is given in
this section), because it does not appear within its
definition. Hence, \code{complex} also does \emph{not} implement the
\code{INumeric} interface. In particular, the \code{complex} type does
not support, i.e. ``implement'', the relational operators \code{<} and
\code{>} that are required for conformance to this interface, given
that these operators are implemented by the \code{integer} and
\code{real(real64)} member types.

%Another way to formulate the above relation is to say that the
%interface is the dual (or intersection) type \cite{Pierce_91}, of the
%type union that it contains. It is this dual view, that makes it
%possible to easily select multiple types for potential use, while
%still being able to rigorously check for conformance to only a single
%type (namely the interface dual of the set), and to thereby enforce
%the idea of ``strong concepts''.

\subsubsection{Kind parameters}

Expanding on the previous example, an \code{INumeric} interface that is
potentially even more useful as a generics constraint could be coded
as follows:
\begin{lstlisting}[language=LFortran,style=boxed]
   abstract interface :: INumeric
      integer(*) | real(*) | complex(*)
   end interface
\end{lstlisting}
Notice how this makes use of both unions of types, and kind parameters
for types, to include all \code{integer}, \code{real}, and
\code{complex} types, that are admitted by the language, in a single
\code{abstract interface} constraint.

The use of kind parameters here is merely syntactic sugar that allows
one to avoid having to write out a type set for all the involved
kinds of a type. For instance, if the particular Fortran
implementation supports \code{real(real32)} and \code{real(real64)} as
its only \code{real} types, then \code{real(*)} is understood to mean
the type set ``\code{real(real32) | real(real64)}''. Notice, also,
that the more types are added to an interface in this fashion, the
smaller the set of intersecting methods will usually become.

\subsubsection{Empty interface}

In the limit of adding all possible types to a type set, there won't
be any common methods left that are implemented by all its types. This
results in the important case of the empty interface, that matches all
types (since any type has at least zero methods):
\begin{lstlisting}[language=LFortran,style=boxed]
   abstract interface :: IAnyType
   end interface
\end{lstlisting}

\subsubsection{Implicit notation}

For simple use cases, it should be optionally possible for the
programmer to employ a shorter notation for declaring type constraints
for generics, than having to define an explicit interface of type
sets, like \code{INumeric} above, and to then have to use it as in
Sect.~\ref{sect:generic_interfaces}. The following modification of
interface \code{ISum}'s original declaration of
Sect.~\ref{sect:generic_interfaces}, provides such an example:
\begin{lstlisting}[language=LFortran,style=boxed]
   abstract interface :: ISum
      function sum{integer | real(real64) :: T}(self,x) result(s)
         import; implicit none
         class(ISum), intent(in) :: self
         type(T),     intent(in) :: x(:)
         type(T)                 :: s
      end function sum
   end interface
\end{lstlisting}
The notation within the generic type parameter list in curly braces
defines a type set interface implicitly, to be used as a type
constraint for type \code{T}. In this particular case, to admit only
the default \code{integer}, or \code{real(real64)} types, for
\code{T}, as discussed above.

\subsubsection{Predefined constraints}

The facilities described in this section are flexible enough for the
user to be able to construct generic constraints himself the way he
needs them. Nevertheless, the language should ideally also supply a
collection of predefined, commonly used generic constraints in the
form of abstract interfaces that are contained in a language intrinsic
module, tentatively called \code{generic\_constraints} here. The list
of such predefined interfaces could include
\begin{itemize}
\item
  an empty interface of the name \code{IAnyType} (as shown above),
\item
  some predefined numeric interfaces allowing for different
  numeric operations, but also
\item
  some predefined interfaces to allow for the use of relational
  operators with different types.
\end{itemize}

Such interfaces could then be imported from user code through a
\code{use} statement like in the following example that assumes the
existence of a language defined interface \code{INumeric}:
\begin{lstlisting}[language=LFortran,style=boxed]
module user_code

   use, intrinsic :: generic_constraints, only: INumeric

   abstract interface :: ISum
      function sum{INumeric :: T}(self,x) result(s)
         import; implicit none
         class(ISum), intent(in) :: self
         type(T),     intent(in) :: x(:)
         type(T)                 :: s
      end function sum
   end interface

end module user_code
\end{lstlisting}

\subsubsection{Present limitations}

Interfaces that are formulated in terms of type sets are presently
\emph{exclusively} intended for use as generics constraints. Hence, a
compiler must ensure that they are not used for any other purpose. In
particular, they are not intended to be implemented by derived
(i.e. user-defined) types\footnote{One of the problems here is that
any new intrinsic function that would have to be added to the language
for some intrinsic type, would change the set of methods of all the
type sets of which this type is a member. This would break any
user-defined types that would implement interfaces which are based on
these type sets.}, and they cannot be used in variable declarations
that involve the \code{class} specifier
(Sect.~\ref{sect:class_specifier_constraints}).

In a future language revision such interfaces could, however, be
admitted for use with the \code{type} declaration specifier, in order
to enable compile-time polymorphism through union (also called sum)
types \cite{Taylor_21,Pierce_91} (which would take the present design
to its logical conclusion, offer an alternative to generic parameters
for certain use cases, and fill a present gap in the later to be
discussed Table~\ref{tab:dispatch}).


\section{Built-in facility for conversion to generic types}

A language like Fortran, that is intended for numeric use, where
conversions between different numeric types are required rather
frequently, must allow conversions to generic types to be done as
easily as it is the case in the Go language. By means that are built
into the language, without having to rely exclusively on external
library functionality.

For instance, generic routines will often have to initialize the
result of reduction operations, as it is, e.g., the case in the test
problem implementation of Sect.~\ref{sect:poly_functional}. There, a
reduction variable for summation, \code{s}, needs to be initialized to
the zero constant of type \code{T}. In Go, it is easily possible to
express this initialization by transforming the (typeless) zero
constant, \code{0}, into the corresponding constant of type \code{T},
i.e. by simply writing \code{s = T(0)}.

If, for instance, \code{T} is then instantiated at compile time with
the \code{float64} type, the expression \code{T(0)} will be
transformed by the compiler into \code{float64(0)}, i.e. a call to the
correct conversion function. In Fortran's case, the compiler would
have to translate the above expression into the intrinsic function
call \code{real(0,kind=real64)}, which should actually be easy to do,
also for all other cases where such conversion is indeed
possible. Otherwise, the compiler should emit an error message, and
abort compilation at the generics instantiation step.

\section{Generic type parameters}
\label{sect:generic_parameters}

As already mentioned above, Fortran's basic generics design should
allow both ordinary and type-bound procedures (i.e. methods), and
derived types to be given their own generic type parameters.

\subsection{Procedures}
\label{sect:generic_procedures}

Using the syntax proposed in this document, an implementation of a
Fortran function for array summation that is generic over the type of its
input array argument would look as follows:
\begin{lstlisting}[language=LFortran,style=boxed]
   function sum{INumeric :: T}(x) result(s)
      type(T), intent(in) :: x(:)
      type(T)             :: s
      integer :: i
      s = T(0)
      do i = 1, size(x)
         s = s + x(i)
      end do
   end function sum
\end{lstlisting}
Generic subroutines can be coded completely analogously.

To actually use the \code{sum} generic function, one simply needs to
call it with arguments of (different) ``numeric'' types (in its
regular arguments list), e.g. as in the two following examples:
\begin{lstlisting}[language=LFortran,style=boxed]
  integer_total = sum([1,2,3,4,5])
  float_total   = sum([1.d0,2.d0,3.d0,4.d0,5.d0])
\end{lstlisting}
Here, the compiler will automatically instantiate appropriate versions
of the \code{sum} generic function by using type inference on the
regular arguments list. Alternatively, generic instantiation can be
accomplished manually by the user, by the additional provision of a
generic type argument in curly braces, as in the following two calls:
\begin{lstlisting}[language=LFortran,style=boxed]
  integer_total = sum{integer}([1,2,3,4,5])
  float_total   = sum{real(real64)}([1.d0,2.d0,3.d0,4.d0,5.d0])
\end{lstlisting}
Such manual instantiation of generic procedures can be useful for
combination with \code{associate} statements, or for use with
procedure pointers, as in the following example:
\begin{lstlisting}[language=LFortran,style=boxed]
  procedure(sum_real64), pointer :: sumf

  abstract interface
     function sum_real64(x) result(s)
        real(real64), intent(in) :: x(:)
        real(real64)             :: s
     end function sum_real64  
  end interface
  
  sumf => sum{real(real64)}

  total1 = sumf([1.d0,2.d0,3.d0,4.d0,5.d0])
  total2 = sumf([2.d0,4.d0,6.d0,8.d0])
\end{lstlisting}

\subsection{Methods}
\label{sect:generic_methods}

The same summation algorithm as that of
Sect.~\ref{sect:generic_procedures}, when implemented as a generic
method, \code{sum}, that is bound to a derived type named
\code{SimpleSum}, which implements the interface \code{ISum} as it was
given in Sect.~\ref{sect:generic_interfaces}, would instead look as
follows:
\begin{lstlisting}[language=LFortran,style=boxed]
module simple_library
   ...

   type, public, implements(ISum) :: SimpleSum
   contains
      procedure :: sum
   end type SimpleSum

contains
   
   function sum{INumeric :: T}(self,x) result(s)
      class(SimpleSum), intent(in) :: self
      type(T),          intent(in) :: x(:)
      type(T)                      :: s
      integer :: i
      s = T(0)
      do i = 1, size(x)
         s = s + x(i)
      end do
   end function sum

end module simple_library
\end{lstlisting}

Generic methods are used completely analogously to generic
procedures. The automatic instantiation use case of
Sect.~\ref{sect:generic_procedures} would, for instance, look as
follows:
\begin{lstlisting}[language=LFortran,style=boxed]
  type(SimpleSum) :: simple

  integer_total = simple%sum([1,2,3,4,5])
  float_total   = simple%sum([1.d0,2.d0,3.d0,4.d0,5.d0])
\end{lstlisting}

\subsection{Derived types}
\label{sect:generic_derived_types}

In addition to procedures, and methods, generic type parameter lists
must also be allowed for derived type definitions, as in the following
example, in which the same interface \code{ISum} from above is
implemented by another derived-type, named \code{PairwiseSum} (here,
using an \code{implements} \emph{statement}):
\begin{lstlisting}[language=LFortran,style=boxed]
module pairwise_library
   ...   
   type, public :: PairwiseSum{ISum :: U}
      private
      type(U) :: other
   end type PairwiseSum

   implements ISum :: PairwiseSum{ISum :: U}
      procedure :: sum
   end implements PairwiseSum

contains

   function sum{INumeric :: T}(self,x) result(s)
      class(PairwiseSum{U}), intent(in) :: self
      type(T),               intent(in) :: x(:)
      type(T)                           :: s
      integer :: m
      if (size(x) <= 2) then
         s = self%other%sum(x)
      else
         m = size(x) / 2
         s = self%sum(x(:m)) + self%sum(x(m+1:))
      end if
   end function sum

end module pairwise_library
\end{lstlisting}
Type \code{PairwiseSum} depends on a generic type parameter \code{U},
that is used within \code{PairwiseSum} itself in order to declare a
field variable of \code{type(U)}, which is named \code{other}. As is
indicated by the type constraint on \code{U}, object \code{other}
conforms to the \code{ISum} interface itself, and therefore contains
its own implementation of the \code{sum} procedure.

The above example furthermore demonstrates, how a derived type's
generic parameters are brought into the scope of its type-bound
procedures via the latters' passed-object dummy arguments. In this
example, type \code{PairwiseSum}'s method, \code{sum}, has a
passed-object dummy argument, \code{self}, that is declared being of
\code{class(PairwiseSum\{U\})}. Hence, method \code{sum} can now
access \code{PairwiseSum}'s generic parameter \code{U}. This
allows the method to make use of two independently defined generic
type parameters, \code{T} and \code{U}, which grants it increased
flexibility. This also means that there is \emph{no} implicit
mechanism of bringing generic parameters of a derived type into the
scope of its methods. If a type-bound procedure needs to access the
generic parameters of its derived type, it must be provided with a
passed-object dummy argument.

Notice, finally, that the declaration \code{class(PairwiseSum\{U\})}
does not imply any ambiguities or contradictions with respect to
compile-time vs. run-time polymorphism, because substitution semantics
apply. At compile time, the compiler will substitute a set of
different type arguments for the generic parameter \code{U}. Hence,
the notation \code{PairwiseSum\{U\}} really refers to a set of
multiple, related, but \emph{different} \code{PairwiseSum} types,
whose \emph{only} commonality is that they all implement the
\code{ISum} interface (and furthermore contain different field
components that do the same). Of course, passed-object dummy arguments
of any of the different \code{PairwiseSum} types of this set can then
be run-time polymorphic, in exactly the same manner that passed-object
dummy arguments of other derived types that implement the same
interface can be run-time polymorphic.


\section{Updated structure constructors}
\label{sect:param_constructors}

If a derived type is parameterized over a generic type, as in the
example of Sect.~\ref{sect:generic_derived_types}, then its structure
constructor must also be assumed to be parameterized over the same
generic type. Hence, calls of structure constructors that are
instantiated with specific argument types substituting the generic
type parameters of their derived types, like in the following
initialization of an object called \code{drv},
\begin{lstlisting}[language=LFortran,style=boxed]
 drv = PairwiseSum{SimpleSum}()
\end{lstlisting}
must be legal. As in Sect.~\ref{sect:generic_methods},
\code{SimpleSum} would be a derived type that implements the
\code{ISum} interface, but (in contrast to the \code{PairwiseSum}
type, as implemented in Sect.~\ref{sect:generic_derived_types}) is not
parameterized by any generic type parameter itself.

We also propose to make a further, small, but important addition to
structure constructors: namely to introduce the notion that a
structure constructor is implicitly defined within the same scope that
holds the definition of its derived type. Assuming that this scope is
a module, then the structure constructor will always be able to access
all the components of its derived type, even if these are declared
being \code{private} to the module's scope. Hence, these components
could be initialized even by calls to the structure constructor that
are being performed from outside the module's scope, in complete
analogy to how user-defined constructors work in Fortran.

In this way, it would become possible to initialize \code{private},
\code{allocatable} derived type components by structure constructors,
something that is crucial for concise OO programming. Since such an
extension would merely add to the capabilities of the language, it
would be fully backwards compatible. The elegance of the afore-given
Go and Swift code versions, but also of the Fortran code examples that
are presented in the next chapter is largely due to the use of such
enhanced structure constructors. Lacking these, user-defined
constructors would have to be used, leading to overly complex
implementations, as e.g. in the Rust example code given in
Listing~\ref{lst:OORust}.


\section{Associated types}

It is often the case, that a method signature needs to declare
arguments that are of exactly the same generic type(s), over which its
own derived type was parameterized. This problem can, in principle, be
solved by making use of generic type parameters in both the derived
type \emph{and} its method, and then enforcing the consistency of
these parameters, by passing the same type arguments to them during
instantiation. A better solution, though, is to allow ``associated
types'' to be declared within abstract interfaces, as it is possible
in both the Rust and Swift languages.

The following example illustrates how an associated type is declared
inside an abstract interface, \code{IAppendable}, that requires any
implementing type to provide functionality for appending items to
itself. These items are declared here to be of \code{type(Element)},
where \code{Element} is an alias, or placeholder, for any actual types
that will be provided by implementations of that interface.

\lstinputlisting[language=LFortran,style=boxed]{Code/Fortran/vector.f90}

An example of such an implementation of interface \code{IAppendable}
is provided here by the derived type \code{Vector}, that stores an
\code{elements} array of generic \code{type(U)}, where the type
parameter \code{U} conforms to the \code{IAnyType} constraint. Notice,
that in order to maintain consistency between the type of the
\code{elements} array, and any new item that we wish to append to it,
we must force the equally named \code{item} argument of method
\code{append} to be of \code{type(U)} as well, as it is shown in this
method's actual implementation. In order to accomplish this
enforcement without contradicting interface \code{IAppendable}'s
definition (that doesn't know anything about type \code{U}), we made
use of the placeholder (i.e. associated) type \code{Element} in the
latter interface. Given method \code{append}'s implementation, the
compiler will then implicitly assign type \code{U} to the type alias
\code{Element}, in order to conform to interface \code{IAppendable}'s
definition.

\section{Extensibility to rank genericity}

Fortran's special role, as a language that caters to numeric
programming, demands that any generics design for the language must
allow for the possibility to also handle genericity of array rank. The
present design offers a lot of room in this respect, but since this is
a largely orthogonal issue, and since we consider a discussion of such
functionality to be non-essential for the purpose of a very first
prototype implementation of the generics features described here, we
defer it to a separate document.


\chapter{Fortran versions of the test problem}

In order to comprehensively illustrate how the new features, that were
discussed in the last two chapters, would be used in practice, we will
give in the present chapter both complete functional and encapsulated
(i.e. OO) Fortran code versions of the test problem that was used in
our language survey.

\section{Functional versions}

Fortran presently lacks support for advanced functional programming
capabilities, like closures and variables of higher-order functions,
that are, e.g., available in Go and the other modern languages. In
contrast to the Go code version of Sect.~\ref{sect:poly_functional},
the functional Fortran code versions that are presented in this
section therefore make no attempt to eliminate rigid dependencies on
user-defined function implementations, and content themselves with
demonstrating how the new generics features can be used to eliminate
rigid dependencies on language intrinsic types.

\subsection{Automatic instantiation of generic procedures}

Listing~\ref{lst:Func1Fortran} shows a straightforward generic
functional implementation of the test problem, that uses automatic
type inference by the compiler. The following additional remarks
should help to better understand this code:
\begin{itemize}
\item
  To express type genericity for the arguments and return values of our
  different generic functions, we make use of a type constraint
  expressed by the interface \code{INumeric}, that is implemented as the
  type set \code{integer | real(real64)}.
\item
  Interface \code{INumeric} is defined by the user himself. Thus,
  there is no need for an external dependency.
\item
  Any required conversions to generic types are done using explicit
  casts, as in Go.
\item
  All the required instantiations of generic procedures are done
  automatically by the compiler, based on type inference of the
  actual arguments that are passed to these procedures.
\end{itemize}

\lstinputlisting[language=LFortran,style=boxed,label={lst:Func1Fortran},caption={Fortran version of the array averaging problem with
automatic generics instantiation.}]{Code/Fortran/functional1.f90}

The example demonstrates that using the new generics features together
with a functional programming style is easy, that the syntax is
economical, and that type inference by the compiler should be
straightforward and therefore reliable. Hence, we believe that the
generics features described here will place no burden on the
programmer.


\subsection{Manual instantiation of generic procedures}

It is actually possible to make the Fortran code version that was
given in Listing~\ref{lst:Func1Fortran}, resemble the Go code version
of Listing~\ref{lst:polyfuncGo} a bit closer, by having two procedure
pointers stand in, within the \code{select case} statement of the main
program, for the closures that were used in the Go code. As this is a
good example for demonstrating how generics can be instantiated
manually by the programmer, we give in Listing~\ref{lst:Func2Fortran}
an alternative form of the main program of
Listing~\ref{lst:Func1Fortran} that makes use of such manual
instantiation (as it was discussed in
Sect.~\ref{sect:generic_procedures}).

\lstinputlisting[linerange={53-101},language=LFortran,style=boxed,label={lst:Func2Fortran},caption={Main program using procedure pointers and manual generics instantiation.}]{Code/Fortran/functional2.f90}



\section{Encapsulated versions}

The present section will demonstrate that using the new generics
features seamlessly and easily in a modern-day OO programming setting
is one of the great strengths of the present design. The equal
importance that has been placed in this document on both proper
compile-time and run-time polymorphic (i.e. modern-day OO)
capabilities will allow for uniform dependency management of both
language intrinsic and user-defined types in Fortran, that is on par
with the most modern languages.

\subsection{Dynamic method dispatch}

Listing~\ref{lst:OOFortran} shows our encapsulated Fortran code
version of the test problem, that corresponds closest to the code
versions that were presented in Chapter~\ref{sect:survey} for all the
other languages.

\begin{itemize}
\item
  As in all these other versions, three interfaces are used to manage
  all the source code dependencies in the problem: \code{INumeric},
  \code{ISum}, and \code{IAverager}. Interface \code{INumeric} is
  defined by the user himself as a type set, similar to the
  corresponding Go code.
\item
  In contrast to the Go and Rust versions (Listings~\ref{lst:OOGo}
  and \ref{lst:OORust}), none of the aforementioned interfaces is
  parameterized itself, since we followed Swift's basic model of
  generics.
\item
  Interface inheritance is expressed through the presence of the
  \code{implements(...)} specifier in a derived-type definition
  (equivalent to Swift).
\item
  The example code makes use, in the main program, of the new
  structure constructors, with their enhancements that were discussed
  in Sect.~\ref{sect:param_constructors}, for the classes
  \code{Averager}, \code{SimpleSum}, and \code{PairwiseSum}.
\item
  This Fortran version makes use of modules and \code{use} statements
  with \code{only} clauses, in order to make explicit the source code
  dependencies of the different defined classes.
\end{itemize}

\lstinputlisting[language=LFortran,style=boxed,label={lst:OOFortran},caption={Proposed encapsulated Fortran version of the array averaging example.}]{Code/Fortran/mixed.f90}

The most important point to notice in Listing~\ref{lst:OOFortran} is
how the main program is the only part of the code that (necessarily)
depends on implementations. The \emph{entire} rest of the code depends
merely on (user-defined) abstract interfaces (see the \code{use}
statements in the above modules). The features described in this
document have enabled us to avoid rigidity in the program, by both
decoupling it and making it operate on multiple intrinsic data types,
thus allowing for a maximum of code reuse. Notice, also, that (as in
the functional Fortran implementation of
Listing~\ref{lst:Func1Fortran}) not a single manual instantiation of
generics is necessary anywhere in the code. The Fortran version
described here is therefore as clean as the Go implementation of
Listing~\ref{lst:OOGo} with respect to dependency management, and it
is as easy to use, and to read and understand, as the Swift
implementation of Listing~\ref{lst:OOSwift}.

\subsection{Static method dispatch}

One of the greatest benefits of the present design is that methods in
OO programming can both be polymorphic, \emph{and} be dispatched
statically (as opposed to dynamically). This will enable inlining of
polymorphic methods by the compiler, to potentially improve code
performance. Listing~\ref{lst:staticFortran} gives a \emph{minimally}
changed version of the encapsulated Fortran code of
Listing~\ref{lst:OOFortran}, to effect static dispatch of the
different \code{sum} methods.

The required changes are confined to a parameterization of the
\code{PairwiseSum} and \code{Averager} derived types, by generic type
parameters that are named \code{U}. These type parameters are then
used in order to declare the field objects \code{other} and \code{drv}
of these derived types, respectively, by means of the \code{type}
specifier. This signifies compile-time polymorphism for these objects
to the compiler (and hence static dispatch of their methods, as
opposed to the \code{class} specifier that was used previously in
order to effect run-time polymorphism and dynamic dispatch. See also
Table~\ref{tab:dispatch} for a summary of rules regarding method
dispatch).

Everything else, especially the declaration of these object variables
as \code{alloctable}s and their instantiation at run-time using
constructor chaining, was kept the same in order to demonstrate that
static method dispatch does \emph{not} necessarily mean that the
actual object instances that contain the methods must be initialized
and their memory allocated at compile-time (although in this
particular case this is possible, as demonstrated in the next section,
given that these objects do not contain any other \code{allocatable}
data fields, like arrays). Notice also that none of the source code
dependencies in the \code{use} statements have changed, i.e. the code
is still fully decoupled, despite making now use of static dispatch.

\lstinputlisting[language=LFortran,style=boxed,label={lst:staticFortran},caption={Demonstrates static method dispatch for the \code{sum} methods.}]{Code/Fortran/static1.f90}

\begin{table}[h]
\centering
\begin{tabular} {L{0.308\textwidth} C{0.308\textwidth} C{0.308\textwidth}}
  \toprule
  \textbf{object declaration} & \textbf{dynamic dispatch} & \textbf{static dispatch} \\
  \midrule
  \code{class(Interface)}   &  always             & never \\
  \code{class(DerivedType)} &  if \code{DerivedType} is \code{extend}ed &
  if \code{DerivedType} is un\code{extend}ed\tablefootnote{Or the method is declared as \code{non\_overridable}.} \\
  \code{type(DerivedType)}  &  never              & always \\
  \code{type(Interface)}  &  presently undefined  & presently undefined \\
  \bottomrule
\end{tabular}
\caption{Correspondence between object declarations and method
  dispatch strategies.}
\label{tab:dispatch}
\end{table}

As in corresponding Rust and Swift implementations of the test
problem, the instantiation of the objects \code{other} and \code{drv}
through constructor calls in the main program has the benefit that the
compiler should be able to infer the correct type arguments that are
required to automatically instantiate the involved generic derived
types. Notice also, that in Listing~\ref{lst:staticFortran}, the
\code{av} object of \code{IAverager} type still needs to make use of
run-time polymorphism, because it is initialized in a \code{select
  case} statement by the main program. This object cannot be made to
employ compile-time polymorphism, as it is initialized within a
statement that performs a run-time decision.

As a final remark we'd like to emphasize that, on readability grounds,
a coding style as that given in Listing~\ref{lst:OOFortran} is
generally preferable over that of
Listing~\ref{lst:staticFortran}. The use of numerous generic type
parameters can quickly make code unreadable. We'd therefore recommend
the default use of run-time polymorphism for managing dependencies on
user-defined types, and the employment of generics for this latter
task only in cases where profiling has shown that static dispatch
would significantly speed up a program's execution (by allowing method
inlining by the compiler). Of course, the use of generics to manage
dependencies on language intrinsic types remains unaffected by this
recommendation.

\subsection{Static method dispatch and static object declarations}

In the present simple example, and taking Listing~\ref{lst:staticFortran}
as a baseline, it is actually possible to even avoid some of the
run-time memory allocation overhead of the program, by having the
field objects \code{other} and \code{drv}, that are contained within
the \code{PairwiseSum} and \code{Averager} types, be statically
declared. To accomplish this, the two lines
\begin{lstlisting}[language=LFortran,style=boxed]
  type(U), allocatable :: other
  type(U), allocatable :: drv
\end{lstlisting}
in Listing~\ref{lst:staticFortran} need to be replaced by the
following two code lines:
\begin{lstlisting}[language=LFortran,style=boxed]
  type(U) :: other
  type(U) :: drv
\end{lstlisting}

Notice that, because the compiler should be able to automatically
infer generic type arguments from the types of regular arguments in
constructor calls, the object instantiations that are to be carried
out from the main program can remain the same, i.e.:
\begin{lstlisting}[language=LFortran,style=boxed]
  avs = Averager(drv = SimpleSum())
  avp = Averager(drv = PairwiseSum(other = SimpleSum()))
\end{lstlisting}
Alternatively, the two calls of \code{Averager}'s structure
constructor can be manually given generic type arguments to confirm
the types of the regular arguments (deleting here their keywords):
\begin{lstlisting}[language=LFortran,style=boxed]
  avs = Averager{SimpleSum}(SimpleSum())
  avp = Averager{PaiwiseSum{SimpleSum}}(PairwiseSum(SimpleSum()))
\end{lstlisting}
Or one can delete the regular arguments to the constructor
altogether, and provide only the generic type arguments, as follows
(see Sect.~\ref{sect:param_constructors}):
\begin{lstlisting}[language=LFortran,style=boxed]
  avs = Averager{SimpleSum}()
  avp = Averager{PairwiseSum{SimpleSum}}()
\end{lstlisting}
The full source code for this version of the test problem can be found
in the \code{Code} subdirectory that is accompanying this document.

\bibliographystyle{plain}
\bibliography{traits}

\renewcommand{\abstractname}{Acknowledgements}

\begin{abstract}
We thank Robert Griesemer of the Go language team for providing the
original code version from which Listing~\ref{lst:polyfuncGo} was
derived, and for his and the Go team's inspirational work on type sets
in Go generics, on which a good fraction of the present design for
Fortran is based. In the same vein, we also thank the many developers
of the Swift, Rust, and Carbon languages who, through their work, have
also influenced the present design.
\end{abstract}

\end{document}
