\documentclass[11pt,oneside]{article}
\usepackage[a4paper, total={6.5in, 9in}]{geometry}
\usepackage{natbib}
\usepackage{url}
\usepackage{listings}
\usepackage{listings-rust}
\usepackage[scaled=0.82]{beramono}
\usepackage[T1]{fontenc}
\usepackage{hyperref}
\usepackage{xcolor}
\usepackage[title]{appendix}
\usepackage{csquotes}
\usepackage{mathpazo}
%\usepackage{mathptmx}

\newcommand{\code}[1]{{\selectfont\ttfamily{#1}}}

\lstdefinelanguage{LFortran}[]{Fortran}{
  morekeywords={abstract,type}
}

\hypersetup{
    colorlinks,
    linkcolor={green!40!black},
    citecolor={blue!70!black},
    urlcolor={blue!90!black}
}

\frenchspacing

\begin{document}

\title{A trait system for the uniform expression of run-time
       and compile-time polymorphism in Fortran}

\author{Konstantinos Kifonidis, Ondrej Certik, Derick Carnazzola}

\maketitle

\abstract{TBD.}

\section{Introduction}

Polymorphism was discovered in the 1960s by Kristen Nygaard and
Ole-Johan Dahl during their development of Simula~67, the world's
first object-oriented (OO) language \cite{Dahl_04}. Their work
introduced into programming what is nowadays known as ``virtual method
table'' (i.e. function-pointer) based run-time polymorphism, which is
both the first focus of this document, and the defining feature of all
OO languages. Several other forms of polymorphism are known today, the
most important of them being parametric polymorphism
\cite{Cardelli_Wegner_85} (also known as ``generics''), which is the
second focus of this document, and which has historically developed
disjointly from run-time polymorphism, since it makes use of
compile-time mechanisms.


\subsection{The purpose of polymorphism}

But what is the purpose of polymorphism in a programming language?
What is polymorphism actually good for? One of the more comprehensive
answers to this question was given by Robert C. Martin in numerous
books (e.g. \cite{Martin_17}), as well as in the following quotation
from his blog:

\begin{displayquote}
``There really is only one benefit to polymorphism; but it's a big
  one. It is the inversion of source code and run time
  dependencies. In most software systems when one function calls
  another, the runtime dependency and the source code dependency point
  in the same direction. The calling module depends on the called
  module. However, when polymorphism is injected between the two there
  is an inversion of the source code dependency. The calling module
  still depends on the called module at run time. However, the source
  code of the calling module does not depend upon the source code of
  the called module. Rather both modules depend upon a polymorphic
  interface. This inversion allows the called module to act like a
  plugin. Indeed, this is how all plugins work.''
\end{displayquote}

Notice, the absence of the words ``code reuse'' in these statements.
The purpose of polymorphism, according to Martin, is the ``inversion''
(i.e. replacement, or management) of source code dependencies by
(means of) particular abstractions, i.e. polymorphic interfaces (or
protocols/traits, as they are also known today). The possibility to
reuse code is then merely the logical consequence of such proper
dependency management.

\subsection{Source code dependencies}

Which then are the source code dependencies that polymorphism helps us
manage? It has been customary to make the following distinction when
answering this question:
\begin{itemize}
\item
 Firstly, most larger programs have dependencies on
 \emph{user-defined} procedures and data types. If the programmer
 employs encapsulation of both a program's procedures and its data,
 i.e. its state, both these dependencies can actually be viewed as
 dependencies on user-defined abstract data types, i.e. types that
 contain both user-defined state, and implementation code which
 operates on that (hidden) state. These are the dependencies that
 Martin is concretely referring to in the above quotation, and it is
 these dependencies on (volatile) implementation (details) that are
 particularly troublesome, because they lead to rigid coupling between
 the various different \emph{parts} of an application. Their results
 are recompilation cascades, the non-reusability of higher-level
 modules, the impossibility to comprehend a large application
 incrementally, and fragility of such an application as a whole.
\item
 Secondly, every program also has dependencies on abstract data types
 that are provided by the programming language in which it is written.
 Fortran's \code{integer}, \code{real}, etc. intrinsic types are
 examples of language intrinsic abstract data types. While hard-wired
 dependencies on such intrinsic types may not couple different parts
 of a program (because the implementations of these types are supplied
 by the language), they nevertheless make a program's source code
 rigid with respect to the data that it can be used on.
\end{itemize}

The most widely used approaches to manage dependencies on language
intrinsic types have so far been through generics, while dependency
management of user-defined (abstract data) types has so far been the
task of OO programming and OO design patterns. Martin \cite{Martin_17}
has, for instance, defined object-orientation as follows:

\begin{displayquote}
  ``OO is the ability, through the use of polymorphism, to gain
  absolute control over every source code dependency in [a software]
  system. It allows the architect to create a plugin architecture, in
  which modules that contain high-level policies are independent of
  modules that contain low-level details. The low-level details are
  relegated to plugin modules that can be deployed and developed
  independently from the modules that contain high-level policies.''
\end{displayquote}

\subsection{Modern developments}

Notice how Martin's modern definition of object-orientation, that
emphasizes source code decoupling, is the antithesis to the usually
taught ``OO'' approaches of one class rigidly inheriting
implementation code from another. Notice also how his definition does
not require some specific type of polymorphism for the task of
dependency management, as long as (according to Martin's first
quotation) the mechanism is based on polymorphic interfaces.

Martin's statements on the purpose of both polymorphism and OO simply
reflect the two crucial developments that have taken place in these
fields over the last decades. Namely, the realizations that
\begin{itemize}
\item
  run-time polymorphism should be freed from the conflicting concept
  of implementation inheritance (to which it was originally bound
  given its Simula~67 heritage), and be formulated exclusively in
  terms of conformance to polymorphic interfaces, i.e. function
  signatures, or purely procedural abstractions, and that
\item
  compile-time polymorphism should be formulated in exactly the same
  way as well.
\end{itemize}

These two developments taken together have recently opened up the
possibility to treat polymorphism, and hence the dependency management
of both user-defined and language intrinsic types, uniformly in a
programming language. As a consequence, it has become possible to use
the potentially more efficient (but also somewhat less flexible)
mechanism of compile-time polymorphism also for a number of tasks that
have traditionally been reserved for run-time polymorphism (i.e. OO
programming), and to mix and match the two polymorphism types inside a
single application to better satisfy a user's needs for both
flexibility and efficiency.

\subsection{Historical background}

The road towards these realizations has been surprisingly long. Over
the last five decades, a huge body of OO programming experience first
had to demonstrate that the use of (both single and multiple)
implementation inheritance breaks encapsulation in OO languages, and
therefore results in extremely tightly coupled, rigid, fragile, and
non-reusable code. This led to an entire specialized literature on OO
design patterns, that aimed at avoiding or mitigating the effects of
such rigidity, by replacing the use of implementation inheritance with
the means to formulate run-time polymorphism that are discussed
below. It also led to the apprehension that implementation inheritance
(but \emph{not} run-time polymorphism) should be abandoned. In modern
languages, implementation inheritance is either kept solely for
backwards compatibility reasons (e.g. in the Swift language), or it is
foregone altogether (e.g. in Rust, Go, and Carbon).

The first statically typed mainstream programming language that
offered a proper separation of run-time polymorphism from
implementation inheritance was Objective~C. It introduced
``protocols'' (i.e. polymorphic interfaces) in the year
1990. Protocols in Objective C consist of pure function signatures,
that lack implementation code. Objective~C provided a mechanism to
implement (multiple) such protocols by a class, and to thus make
classes conform to protocols. This can be viewed as a restricted form
of (multiple) inheritance, namely inheritance of object
\emph{specification}, which is also known as \emph{subtyping}. Only a
few years later, in 1995, the Java language hugely popularized these
ideas under the monikers ``interfaces'' and ``interface
inheritance''. Today, nearly all modern languages support polymorphic
interfaces/protocols/traits, and the mechanism of multiple interface
inheritance that was introduced to express run-time polymorphism in
Objective~C. The only negative exceptions in this respect being modern
Fortran, and C++, which both still stick to the obsolescent Simula~67
paradigm.

A similar learning process, as that outlined for run-time
polymorphism, took place in the field of compile-time/parametric
polymorphism. Early attempts, notably templates in C++, to render
function arguments and class parameters polymorphic, did not impose
any constraints on such arguments and parameters, that could be
checked by C++ compilers. With the known results on compilation times
and cryptic compiler error messages. Surprisingly, Java, the language
that truly popularized polymorphic interfaces in OO programming, did
not provide an interface based mechanism to constrain its
generics. Within the pool of mainstream programming languages, this
latter realization was first made with the advent of Rust
\cite{Matsakis_2014}. Rust came with a trait (i.e. polymorphic
interface) system with which it is possible for the user to uniformly
and transparently express both generics (i.e. compile-time) and
run-time polymorphism in the same application, and to relatively
easily switch between the two, where possible.

Rust's main idea was quickly absorbed by almost all other mainstream
modern languages, most notably Go, Swift, and Carbon, with the
difference that these latter languages tend to leave the choice
between static and dynamic procedure dispatch to the compiler, or
language implementation, rather than the programmer. C++ is in the
process of adopting such constraints for its ``templates'' under the
term ``strong concepts'', but without implementing the greater idea to
uniformly express \emph{all} the polymorphism in the language through
them. An implementation of this latter idea must today be viewed as a
prerequisite in order to call a language design ``modern''. The
purpose of this document is to describe additions to Fortran, that aim
to provide the Fortran language with such modern capabilities.

\newpage

\section{Case Study: Calculating the average value of a numeric array}

To illustrate the advanced features and capabilities of some of the
available modern programming languages with respect to polymorphism,
and hence dependency management, we will make use here of a case
study: the simple test case of calculating the average value of a set
of numbers stored inside a one-dimensional array. In the remainder of
this section we will first provide an account and some straightforward
monomorphic (i.e. coupled) functional implementation of this test
problem, followed by a functional implementation that makes use of
both run-time and compile-time polymorphism to manage source code
dependencies. In the survey of programming languages presented in
Sect.~\ref{sect:survey}, we will then recode this test problem in an
encapsulated fashion, to highlight how the source code dependencies in
this problem can be managed in different languages even in more
complex situations, that require OO techniques.

\subsection{Monomorphic functional implementation}
\label{sect:mono_functional}

We have chosen Go here as a language to illustrate the basic ideas.
Go is easily understood, even by beginners, and is therefore well
suited for this purpose (another good choice would have been the Swift
language). The code in the following Listing~\ref{lst:funcGo} should
be self explanatory for anyone who is even only remotely familiar with
the syntax of C family languages. So, we'll make only a few remarks
regarding syntax.
\begin{itemize}
\item
  While mostly following a C like syntax, variable declarations in Go
  are essentially imitating Pascal syntax, where a variable's name
  precedes the declaration of the type.
\item
  Go has two assignment operators. The usual \code{=} operator, as it
  is known from other languages, and the separate operator \code{:=}
  that is used for combined declaration and initialization of a
  variable.
\item
  Go has array slices that most closely resemble those of Python's
  Numpy (which exclude the upper bound of an array slice).
\end{itemize}

Our basic algorithm for calculating the average value of an array of
integer elements employs two different implementations for
averaging. The first makes use of a ``simple'' summation of all the
array's elements, in ascending order of their array index. While the
second sums in a ``pairwise'' manner, dividing the array in half to
carry out the summations recursively, and switching to the ``simple''
method once subdivision is no longer possible.

As a result, this code has three levels of hard-wired (i.e. rigid)
dependencies. Namely,
\begin{enumerate}
\item
  function \code{pairwise\_sum} depending on function
  \code{simple\_sum}'s implementation,
\item
  functions \code{simple\_average} and \code{pairwise\_average}
  depending on functions' \code{simple\_sum}, and \code{pairwise\_sum}
  implementation, respectively, and
\item
  the entire program depending rigidly on the \code{int} data type in
  order to declare both the arrays that it is operating on, and
  the results of its summation and averaging operations.
\end{enumerate}
The first two items are dependencies on user-defined implementations,
while the third is a typical case of rigid dependency on a language
intrinsic type, which renders the present code incapable of being
applied to arrays of any other data type than \code{int}s. Given that
we are dealing with three levels of dependencies, three levels of
polymorphism will accordingly be required to remove all these
dependencies.

\lstinputlisting[language=Go,style=boxed,label={lst:funcGo},caption={Monomorphic functional version of the array averaging example in Go.}]{Code/Go/coupled.go}

\subsection{Polymorphic functional implementation}
\label{sect:poly_functional}

Listing~\ref{lst:polyfuncGo} gives an implementation of our test
problem, that employs Go's generics and functional features in order to
eliminate the last two of the rigid dependencies that were listed in
Sect.~\ref{sect:mono_functional} (we thank Robert Griesemer of the Go
team for providing the original code of this particular version of the
example). The code makes use of Go's generics to admit arrays of both the
\code{int} and \code{float64} types as arguments to all functions, and
to express the return values of the latter. It also makes use of the
run-time polymorphism inherent in Go's functional features, namely
closures and variables of higher-order functions, to replace the two
previous versions of function \code{average} (that depended on
specific implementations), by a single polymorphic version. Only the
rigid dependency of function \code{pairwise\_sum} on function
\code{simple\_sum} has not been removed, in order to keep the code
more readable. In the OO code versions, that will be presented in
Sect.~\ref{sect:survey}, even this dependency is eliminated.

A few remarks are in order for a better understanding of
Listing~\ref{lst:polyfuncGo}'s code:
\begin{itemize}
\item
  In Go, generic parameters to a function, like the parameter \code{T}
  here, are provided in a separate parameter list, that is enclosed in
  brackets [ ].
\item
  Generic parameters have a constraint that follows their declared
  name. Go uses exclusively interfaces as such constraints (see the
  interface \code{INumeric} in the present example).
\item
  Interfaces consist of either function signatures, or \emph{type
  sets}, like ``\code{int | float64}'' in the present example. The
  latter signify a set of function signatures, too, namely the
  signatures of the intersecting set of all the operations/functions
  for which the listed types provide implementations.
\item
  The code makes use of type conversions to the generic type \code{T},
  where required. For instance, \code{T(0)} converts the
  \code{integer} constant \code{0} to the corresponding zero constant
  of type \code{T}.
\item
  The code instantiates closures and stores these by value in two
  variables named \code{avi} and \code{avf} for later use (Fortran
  and C programmers should note that \code{avi} and \code{avf} are
  \emph{not} function pointers!).
\end{itemize}


\lstinputlisting[language=Go,style=boxed,label={lst:polyfuncGo},caption={Polymorphic functional version of the array averaging example in Go.}]{Code/Go/functional.go}

The motivation to code the example as in Listing~\ref{lst:polyfuncGo}
is that once the two closures \code{avi}, and \code{avf}, are properly
instantiated (by means of the \code{switch} statement), they may be
passed from the main program to any other client code that may need to
make use of the particular averaging algorithm that was selected by
the user. This latter client code would \emph{not} have to be littered
with \code{switch} statements itself, and it would \emph{not} have to
depend on any specific implementations. It would merely depend on the
closures' interfaces. The same holds for the OO code versions that are
discussed in the next section, with objects replacing the closures
(both being merely slightly different realizations of the same idea).

\section{Survey of modern languages}
\label{sect:survey}

In the present section we give implementations, in various modern
languages, of encapsulated (i.e. OO) code versions of the test
example. As in the functional example presented in
Sect.~\ref{sect:poly_functional}, we employ run-time polymorphism to
manage the dependencies on user-defined implementations (in this case
abstract data types), and generics in order to manage the dependencies
on language intrinsic types. This serves to illustrate how both
run-time and compile-time polymorphism can be typically used for
dependency management in an OO setting in these modern languages. The
survey also serves to highlight the many commonalities but also some
of the minor differences in the approaches to polymorphism that were
taken in these different languages. As a final disclaimer, we do not
advocate to code problems in an OO manner that can be easily coded in
these languages in a functional way (as it is the case for this
problem). However, in more complex cases, where many more nested
functions would need to be used, and where state would have to be
hidden, the OO programming style would be the more appropriate
one. Hence our test problem will stand in, in this section, for
emulating also such a more complex problem, that would benefit from an
encapsulated coding style.


\subsection{Go}

Go has supported run-time polymorphism through (polymorphic)
``interfaces'' (and hence modern-day OO programming) since its
inception. In Go, encapsulation is done by storing state in a
``\code{struct}'' and by binding procedures, that need to use that
state, to this same \code{struct}. Thus creating a user-defined
abstract data type (or ADT) with methods. Go allows the programmer to
make such ADTs conform to polymorphic interfaces, by implementing all
the functions whose signatures are contained in an interface.

In Go, there is no explicit statement for the purpose of implementing
interfaces (i.e. for interface inheritance). Instead, an ADT is
implicitly assumed to implement an interface whenever it provides
implementations of all the interface's function signatures. This way
of implementing methods requires only a reference to an ADT object to
be passed to the method (by means of a separate parameter list, in
front of the method's actual name). It is otherwise completely
decoupled from the ADT's (i.e. the actual \code{struct}'s)
definition. Hence, existing ADTs (even if their source code itself is
inaccessible) can be made to retroactively implement new interfaces
and methods, and hence to be used in new settings. Go, finally, makes it
explicit in its syntax that interfaces are types in their own right,
and that hence polymorphic variables (i.e. objects) can be declared in
terms of them.

Since version 1.18, Go also supports compile-time polymorphism through
generics. Go's generics make use of ``strong concepts'', since they
are bounded by constraints that are expressed through
interfaces. Hence, the Go compiler will fully type-check generic code.
In Go, structures, interfaces, and functions (but not methods) can all
be given their own generic type parameters. To allow methods to make
use of such parameters one has to parameterize the structures and
interfaces to which these methods or, respectively, their signatures
belong.

\subsubsection{Encapsulated version coded in Go}

Listing~\ref{lst:OOGo} gives an encapsulated version of the test
example coded in Go. The two versions of the \code{sum} function have
been encapsulated in two different ADTs named \code{SimpleSum} and
\code{PairwiseSum}, whereas a third ADT named \code{Averager}
encapsulates the functionality that is required to perform the actual
averaging. The latter two ADTs contain the lower-level objects
``\code{other}'' and ``\code{drv}'' of \code{ISum[T]} type as
components, to which they delegate calls to these objects' \code{sum}
methods. Notice how the use of the polymorphic interface
\code{ISum[T]} in the respective declarations enables either
\code{SimpleSum} or \code{PairwiseSum} objects to be plugged into
their higher-level clients.

A further interface, named \code{IAverager}, is used to enable
polymorphism for different averaging algorithms. Finally, there's a
third interface, \code{INumeric}, that serves exactly the same purpose
as in the functional polymorphic version given in
Sect.~\ref{sect:poly_functional}, namely to make all function
arguments and return values polymorphic, by admitting as input and
output parameters both the \code{int} and \code{float64} intrinsic
types.

Hence, three polymorphic interfaces were required in this code, in
order to eliminate the three levels of rigid dependencies that were
listed in Sect.~\ref{sect:mono_functional}. Notice also that,
exempting \code{INumeric}, all the interfaces and all the user-defined
ADTs need to take generic parameters in this example. This is required
in order to enable all the \code{sum} and \code{average} methods to
use generic type parameters in Go.

\lstinputlisting[language=Go,style=boxed,label={lst:OOGo},caption={Encapsulated Go version of the array averaging example.}]{Code/Go/mixed.go}

The main program makes use of Go's built-in structure constructors in
order to instantiate objects of all the ADTs. In particular, it
instantiates run-time polymorphic ``\code{Averager}'' objects
(depending on whether simple or pairwise sum averaging is to take
place), and it does so for both the \code{int} and \code{float64}
types separately, in order to then use these objects on \code{int} and
\code{float64} data, respectively. The fact that \code{two} such
objects are required (one for each language intrinsic data type) is
connected to the fact that in order to obtain generic methods in Go,
one has to parameterize interfaces by generic parameters, and
instantiate them with different data types, as in \code{func main}'s
first two code lines. A single (i.e. unparameterized) \code{IAverager}
interface therefore doesn't suffice, which is unfortunate from the
user's perspective, as some code duplication in client code cannot be
avoided in this way.

\subsection{Rust}

Like Go, Rust supports both run-time and compile-time polymorphism
through polymorphic interfaces, which Rust calls ``traits''.

\subsubsection{Encapsulated version coded in Rust}

\lstinputlisting[language=Rust,style=boxed,label={lst:OORust},caption={Encapsulated Rust version of the array averaging example.}]{Code/Rust/mixed_poly/src/main.rs}

\subsection{Swift}

Being a successor language to Objective~C, Swift differs slightly from
the languages considered so far in that it opted to retain
implementation inheritance for backwards compatibility to Objective~C,
whereas both Go and Rust do not support implementation inheritance
\emph{by design}. Swift therefore supports ``classical'' classes, but
it also allows one to bind methods to structures (which, in contrast to
classes, are value types in Swift).

Like Go and Rust, Swift (also) supports run-time and compile-time
polymorphism through polymorphic interfaces, that are called
``protocols'' in Swift. If the Swift programmer chooses to ignore
implementation inheritance and classes, he can therefore very much
program with structures and protocols in Swift as he would with
structures and interfaces/traits in Go and Rust, respectively. We will
demonstrate this in the next section.

Swift generics support ``strong concepts'', and are thus fully
type-checked, and their capabilities are on par with those of Go and
Rust. For instance, all these languages allow for the parameterization
of functions and structures (and in Swift's case also for
classes). Where Swift differs from both Go an Rust, is that it also
allows for parameterized \emph{methods}. Whereas both Go and Rust have
their users parameterize entire interfaces/traits instead, in order to
achieve the same functionality. This has some interesting consequences
for the programmer, that will be discussed in detail below.

\subsubsection{Encapsulated version coded in Swift}

Listing~\ref{lst:OOSwift} gives an example of how the encapsulated
version of the array averaging test problem can be programmed in
Swift. See the following remarks in order to understand this code:

\begin{itemize}
  \item
    Swift uses angled brackets \code{<>} to indicate generic parameter
    lists.
  \item
    Type constraints are formulated by supplying a protocol name after a
    type parameter (separated by a colon).
  \item
    Swift does not supply an equivalent to Go's \code{int | float64}
    syntax. Hence the user must use a \code{Numeric} protocol defined
    by the standard library, as a constraint for numeric types. Which
    leads to reliance on library code.
  \item
    Unfortunately, Swift's \code{Numeric} protocol does \emph{not} support
    the division operation! Hence the division that would have been required
    in function \code{average} of the \code{Averager} struct had to be moved
    out to the calling code of the main program.
  \item
    Implementation of a protocol (i.e. interface inheritance) is
    signified in Swift with the \code{:} operator appearing behind a
    structure's (or a class's) name. There is no separate \code{impl}
    block as in Rust, or implicit conformance to a protocol by
    implementing all its methods, as in Go. Functionality to implement
    protocols retroactively is provided through an ``extension''
    block, instead.
  \item
    The Swift version makes use of language built-in, default, structure
    constructors (called ``initializers'').
  \item
    Array slices are not arrays themselves. So an explicit conversion
    using an \code{Array()} constructor is required in such cases.
  \item
    By default, function and method calls in Swift make use of keyword
    arguments.
  \item
    The syntax for type conversion into a generic type \code{T} is
    somewhat peculiar. E.g. Go's \code{T(0)} is written as
    \code{T(exactly:0)!} in Swift (making use of the mandatory keyword
    ``\code{exactly}'' in the function responsible for the type
    conversion).
\end{itemize}

\lstinputlisting[language=Swift,style=boxed,label={lst:OOSwift},caption={Encapsulated Swift version of the array averaging example.}]{Code/Swift/mixed.swift}

Even a casual glance at the Swift version will show that the Swift
code is the easiest to read and understand among all the OO
implementations. This is largely the result of Swift supporting
generic methods, and hence not requiring the programmer to
parameterize and instantiate any generic interfaces/protocols, in
contrast to both Go and Rust. The consequences are
\begin{itemize}
\item
that method genericity for some ADT can be expressed using only a
single, as opposed to multiple protocols,
\item
that merely a \emph{single} object instance of that same protocol is
required, whose methods are then able to operate on many different
language intrinsic data types, and
\item
that this also largely \emph{obviates the need for manual instantiations
of generics in Swift} (because generic functions/methods are easier to
instantiate automatically by the compiler, as it can always infer the
required types by checking the actual arguments that are passed to a
function/method)!
\end{itemize}

As an example, consider the \code{IAverager} protocol in the above
Swift code. There's only a single (i.e. unparameterized) version of
this protocol. Consequently, there's only a need in the main program
to declare a single object variable, \code{av}, of that protocol (that
enables \code{av} to be polymorphically assigned different
\code{struct}s that implement \code{IAverager}). Because it contains
an ``\code{average}'' method that is generic, this \emph{single}
object can then be straightforwardly used on data of \emph{both} the
\code{Int} and \code{Float64} types!

This vastly simplifies client code that needs to make use of objects
such as \code{av}, especially if such client code needs to work on
\emph{many} more types than just \code{Int} and
\code{Float64}. Contrast this with Go's and Rust's model, where manual
instantiation of a different version of \code{IAverager} is required
for \emph{every} different generic type parameter that the user wishes
to employ. Notice also, how there's \emph{not a single manual
instantiation} of any generics in the Swift code example! We consider
these significant advantages of the generics approach that is taken in
Swift vs. that of Go and Rust.


\subsection{Conclusions from the different implementations}

Since the use of run-time polymorphism by means of interfaces is very
similar in all the languages considered here, we will only make some final
comments on the languages' compile-time, i.e.  generics features.

\subsubsection{Conclusions on Go}

Go's basic model to implement generics allows structures, interfaces,
and ordinary functions, but not methods, to be given their own generic
type parameters. The lack of true generic methods can make some code
duplication in client code unavoidable. Nevertheless, generic Go code
is quite easy to read and to understand. What sets Go apart from some
of the other languages is its built-in, easy to use support for
conversion to generic types, and especially its brilliant new notion
to interpret interfaces as type sets, along with its syntax to support
this notion. This enables the Go programmer to easily tailor
constraints on generic types to his specific needs, which is what
makes the use of generics in Go pleasant.

\subsubsection{Conclusions on Rust}

Rust's basic model for generics is similar to Go's in that it
allows for parameterization of structures, interfaces, and ordinary
functions. Hence, what has been said above for Go holds also for Rust.
Rust has some quirks which render its use for the management of all
types of dependencies through polymorphism somewhat sub-optimal when
compared to the other languages considered here. The language is
unpleasant to use, because of its ``borrow checker'', its employment
of move semantics by default, its excessive obsession with type
safety, and its general C++-like philosophy to copiously rely on
external dependencies, even for the most basic tasks, like
initializing a generic type. The Rust version of our test case is
therefore marred by some dependencies on external libraries (called
``crates'' in Rust), which are quite contrarian to the purpose of
programming in a polymorphic fashion, namely to avoid rigid
dependencies. But even with the functionality provided by such
external dependencies, Rust doesn't allow type conversion to generic
types within generic routines. A necessary capability for numerical
work that is, for instance, built into Go.

\subsubsection{Conclusions on Swift}

Swift's basic model of implementing generics by allowing parameterized
structures and methods (but not parameterized interfaces) is both the
easiest to read, and the easiest to use from a programmer's
perspective. Swift's generics design allows the Swift compiler to
instantiate generics largely automatically, through inspection of the
argument types that are passed to functions and methods. In contrast
to the other languages, in Swift, the user basically never has to
bother with instantiating a generic function, a method, and often not
even a structure or a class.

If the Swift programmer knows how to write generic functions, his
knowledge automatically translates into coding generic methods, since
generic functions can be transformed into generic methods without
requiring any changes to their function signatures. This property is
helpful for the refactoring of non-OO codes into corresponding OO
versions.

We hence consider Swift's generics to be the most attractive model to
base Fortran's basic generic capabilities on, provided that it can be
implemented sufficiently easily. The fact that Swift is a language
that does not put emphasis on numerics, and whose present standard
library therefore does not provide a truly useful \code{Numeric}
protocol (that supports all the usual numeric operations), is of
absolutely no consequence for adopting Swift's basic generics design
for Fortran.

Fortran will necessarily do a better job in this respect, both by
borrowing Go's idea of interpreting type sets as interfaces, so that
the user can easily implement his own type constraints. But also by
making accessible to the user a set of language-built in interfaces
that are truly useful for numeric operations, and are implemented by
Fortran's intrinsic types.


\newpage

\section{Fortran additions I: File extension with new defaults}

\section{Fortran additions II: Subtyping}

\subsection{Named abstract interfaces (traits)}

\subsection{Multiple interface inheritance}

\subsection{Enhanced type declarations}

Named abstract interfaces are types in their own right. Hence,
(polymorphic) instance variables can be declared in terms of them. To
enable the Fortran programmer to do so, Fortran's \code{type}
declaration specifier needs to be extended to accept named abstract
interfaces.

\subsection{Improved structure constructors}


\newpage

\section{Fortran additions III: Generics}

The new features that were discussed in the previous section are
required in order to uniformly express and support both run-time and
compile-time polymorphism in Fortran. We will now proceed with
discussing some additional new features that are exclusively required
in order to further support compile-time polymorphism, i.e. generics.

\subsection{Interfaces containing generic procedure signatures}
\label{sect:generic_interfaces}

Abstract interfaces should be allowed to contain signatures of generic
procedures, as in Swift. Go's and Rust's approach to parameterize
abstract interfaces themselves, appears not as attractive from a
user's perspective. The following code shows, as an example, an
abstract interface named \code{ISum} that contains the signature of a
generic type-bound procedure named \code{sum}:
\begin{lstlisting}[language=LFortran,style=boxed]
   abstract interface :: ISum
      function sum{INumeric :: T}(self,x) result(s)
         type(ISum), intent(in) :: self
         type(T),    intent(in) :: x(:)
         type(T)                :: s
      end function sum
   end interface
\end{lstlisting}

The example illustrates the use of a generic type parameter, i.e.  a
meta-type, or a \emph{type of types}. In this example, this type
parameter is simply called \code{T}, and it is preceded by the name of
an abstract interface that expresses a constraint on the type
parameter. Fortran generics thus support ``strong concepts''. Both,
the type parameter and its constraint, are part of a generic type
parameter list that is enclosed in curly braces, and follows
immediately behind the procedure's name.


Notice that, since \code{T} is a meta-type, there are some significant
differences to types that are specified in the standard parameter
list. For instance, the specification of a rank, or an \code{intent},
for meta-types like \code{T}, makes no sense. This is because the
latter are always scalar input parameters. The syntax used above,
that deviates slightly from how Fortran's usual function arguments are
declared, therefore appears justified as it reflects that, in type
parameters, one is dealing with different entities.

\subsection{Interfaces as type sets}

\begin{lstlisting}[language=LFortran,style=boxed]
   abstract interface :: INumeric
      integer | real(real64)
   end interface
\end{lstlisting}

Different ``length'' parameters for character variables would be
handled in the same fashion. The following interface would, for the
lack of a better example, be used to only admit \code{character}
variables with a length of either 4 or 8 characters:
\begin{lstlisting}[language=LFortran,style=boxed]
   abstract interface :: IPrintable
      character(len=4) | character(len=8)
   end interface
\end{lstlisting}

For simple use cases, it should be optionally possible for the
programmer to employ a shorter notation for declaring type constraints
than having to write a separate interface like \code{IPrintable}, or
\code{INumeric} above. As in the following modification of interface
\code{ISum}'s declaration, that was given previously in
Sect.~\ref{sect:generic_interfaces}:
\begin{lstlisting}[language=LFortran,style=boxed]
   abstract interface :: ISum
      function sum{integer | real(real64) :: T}(self,x) result(s)
         type(ISum), intent(in) :: self
         type(T),    intent(in) :: x(:)
         type(T)                :: s
      end function sum
   end interface
\end{lstlisting}
The above notation would then define an abstract interface implicitly,
to be used as a type constraint for type \code{T}. In this particular
case, to admit only the default \code{integer}, or \code{real(real64)}
types, for \code{T}.

\subsection{Predefined interfaces for expressing common constraints}

The language should ideally supply some predefined, commonly used generic
constraints in the form of abstract interfaces that are contained in a
language intrinsic module. The actual implementation of these interfaces
could then, of course, employ the ``interfaces-as-type-sets'' syntax that
was described above. For instance, a more general \code{INumeric}
interface than the one given above, could be implemented as follows:
\begin{lstlisting}[language=LFortran,style=boxed]
   abstract interface :: INumeric
      integer(*) | real(*) | complex(*)
   end interface
\end{lstlisting}
Notice how this makes use of kind parameters to include all
\code{integer}, \code{real}, and \code{complex} types, admitted by the
language, in a single \code{abstract interface} constraint.
Such language provided, interfaces could then be used from user code
through a \code{use} statement like in the following example for
\code{INumeric}
\begin{lstlisting}[language=LFortran,style=boxed]
module user_code

   use, intrinsic :: generic_constraints, only: INumeric

   abstract interface :: ISum
      function sum{INumeric :: T}(self,x) result(s)
         type(ISum), intent(in) :: self
         type(T),    intent(in) :: x(:)
         type(T)                :: s
      end function sum
   end interface

end module user_code
\end{lstlisting}
for use as a constraint in function and derived type implementations,
or in other interfaces, like \code{ISum} here, that require the
functionality of \code{INumeric}.

\subsection{Conversions to generic types}

\subsection{Generic type parameters for methods and procedures}

As already mentioned above, LFortran's design of generics should
follow Swift's, if possible, and allow both ordinary and type-bound
procedures (i.e. methods) to be given their own generic type
parameters. An implementation of the generic method \code{sum} of
Sect.~\ref{sect:generic_interfaces}, that is bound to a derived type
with name \code{SimpleSum}, would look as follows:
\begin{lstlisting}[language=LFortran,style=boxed]
   function sum{INumeric :: T}(self,x) result(s)
      type(SimpleSum), intent(in) :: self
      type(T),         intent(in) :: x(:)
      type(T)                     :: s
      integer :: i
      s = T(0)
      do i = 1, size(x)
         s = s + x(i)
      end do
   end function sum
\end{lstlisting}
Whereas the next example illustrates how the same procedure would look
as a stand-alone (i.e. non-encapsulated) generic function:
\begin{lstlisting}[language=LFortran,style=boxed]
   function sum{INumeric :: T}(x) result(s)
      type(T), intent(in) :: x(:)
      type(T)             :: s
      integer :: i
      s = T(0)
      do i = 1, size(x)
         s = s + x(i)
      end do
   end function sum
\end{lstlisting}

\subsection{Generic type parameters for derived types}

In addition to procedures, generic type parameter lists must be allowed
also for derived types, as in the following example, in which 
the interface \code{ISum} from above is implemented by a parameterized
derived-type named PairwiseSum:
\begin{lstlisting}[language=LFortran,style=boxed]
   type, implements(ISum) :: PairwiseSum{ISum :: U}
      private
      type(U) :: other
   contains
      procedure :: sum
   end type PairwiseSum
\end{lstlisting}
\code{PairwiseSum} depends on a generic type parameter \code{U}, that
is used within \code{PairwiseSum} itself in order to declare a field
variable of \code{type(U)}, which is named \code{other}. As is
indicated by the type constraint on \code{U}, object \code{other}
conforms to the \code{ISum} interface itself, and therefore contains
its own implementation of the \code{sum} procedure.


\subsection{Generic type parameters for structure constructors}
\label{sect:param_constructors}

If a derived type is parameterized with a generic type, then its
structure constructor must also be assumed to be parameterized with
the same generic type. Hence, calls of structure constructors that are
instantiated with particular argument types replacing the generic type
parameters of their derived types, like e.g.
\begin{lstlisting}[language=LFortran,style=boxed]
   Averager{SimpleSum}()
   Averager{PairwiseSum{SimpleSum}}()
\end{lstlisting}
must be legal. Here, \code{SimpleSum} would be a derived type that
implements the \code{ISum} interface, but (in contrast to the
\code{PairwiseSum} and \code{Averager} types) is not parameterized by
any generic type parameters itself.


\subsection{Extensibility to rank genericity}

Fortran's special role, as a language that caters to numeric
programming, demands that any generics design for the language must
allow for the possibility to also handle genericity of array rank. The
present design offers a lot of room in this respect. But we prefer to
focus on such an extension in a future document, because it is an
issue that is completely orthogonal to the genericity of types
(although it would be handled in much the same way).

\newpage

\section{Proposed Fortran versions of the test example}

\subsection{Functional version}

{\sf (What to do about this one, given that Fortran doesn't have Go's
  functional programming capabilities? In this case, I think it should
  be sufficient to demonstrate only the decoupling of the argument
  types through generics, and leave all the other coupling in, as
  it is the case in the coupled functional Go version.)}

\subsection{Encapsulated version}

Listing~\ref{lst:OOFortran} gives our Fortran version of the
encapsulated form of the test example that corresponds to the code
versions that were presented in Sect.~\ref{sect:survey} for all the
other languages.

\begin{itemize}
\item
  We employ here the Go borrowed syntax \code{integer | real(real64)}
  to implement the interface \code{INumeric}, that is used in order to
  express type genericity for the array \code{x} and the result of the
  summation \code{s} in our different implementations of method
  \code{sum}.
\item
  As in the corresponding Go version, \code{INumeric} is defined by
  the user himself as a type set consisting of the set of intersecting
  operations defined in Fortran for the \code{integer} and
  \code{real(real64)} types.  There is thus no need for an external
  dependency.
\item
  The remaining interfaces \code{ISum} and \code{IAverager} make use
  of generic methods that are declared in terms of \code{INumeric}.
  However, in contrast to the Go version, none of these interfaces is
  parameterized itself, since we followed Swift's model of generics.
\item
  Interface inheritance is expressed through the presence of the
  \code{implements(...)} specifier in a derived-type definition
  (equivalent to Swift).
\item
  Conversions to generic types are done as in Go. Notice, how the
  compiler will have to do the necessary replacements of, e.g.,
  \code{T(0)} in function \code{sum} of class \code{SimpleSum} by
  calls to Fortran's correct conversion functions for integer and real
  types of the right kinds.
\item
  The example code makes use, in the main program, of the new
  structure constructors, with their enhancements that were discussed
  in Sect.~\ref{sect:param_constructors}, for the classes
  \code{Averager}, \code{SimpleSum}, and \code{PairwiseSum}.
\item
  The Fortran version makes use of modules and \code{use} statements
  with \code{only} clauses, in order to make explicit the source code
  dependencies of the different defined classes.
\end{itemize}

\lstinputlisting[language=LFortran,style=boxed,label={lst:OOFortran},caption={Proposed encapsulated Fortran version of the array averaging example.}]{Code/Fortran/mixed.ft}

The most important point to notice in Listing~\ref{lst:OOFortran} is
how the main program is the only part of the code that (necessarily)
depends on implementations. The \emph{entire} rest of the code depends
merely on abstract interfaces (see the \code{use} statements in the
above modules). The Fortran version described here is therefore as
clean as the Go implementation with respect to dependency management,
and as easy to use as the Swift implementation.

The features described in this document have enabled us to avoid
rigidity in the program, by both decoupling it and making it operate
on multiple data types, thus allowing for a maximum of code
reuse.

\subsection{Encapsulated (mostly) static version}

To finally demonstrate how methods can be made to use static, as
opposed to dynamic dispatch, Listing~\ref{lst:staticFortran} gives a
version of the encapsulated Fortran code that contains all the
required changes, as compared to Listing~\ref{lst:OOFortran}, to
effect static dispatch of the \code{sum} methods.

The changes are confined to a parameterization of the
\code{PairwiseSum} and \code{Averager} derived types, by generic type
parameters that are named \code{U}. These type parameters are then
used in order to statically declare the field objects ``\code{other}''
and ``\code{drv}'' of these derived types, respectively, that were
previously declared as \code{allocatable} variables that were to be
initialized dynamically at run-time. Correspondingly, there is no
longer any such dynamic instantiation necessary for these objects. The
only things that are required are instantiations of the parameterized
derived types in which these objects are contained. Notice how these
instantiations are carried out from the main program within the calls
of \code{Averager}'s constructor, which is provided with the types
\code{SimpleSum}, and \code{PairwiseSum}.


\lstinputlisting[language=LFortran,style=boxed,label={lst:staticFortran},caption={Encapsulated Fortran version of the array averaging example with static method dispatch.}]{Code/Fortran/static.ft}

Notice also, that in Listing~\ref{lst:staticFortran}, the \code{av}
object of \code{IAverager} type that is initialized in the
\code{select case} statement of the main program still needs to make
use of run-time polymorphism. This object cannot be made to employ
compile-time polymorphism, as it is used within a statement that
performs a run-time decision.

As a final remark we'd like to emphasize that a coding style as that
given in Listing~\ref{lst:OOFortran} should generally be preferred
over that of Listing~\ref{lst:staticFortran}, as it leads to more
readable code. The use of numerous generic parameters can quickly make
code unreadable. We'd therefore recommend the default use of run-time
polymorphism and dynamic dispatch for managing dependencies on user
implementation code, and the employment of generics and static
dispatch for this latter task only in cases where profiling has shown
that static dispatch would significantly speed up a program's
execution (by allowing method inlining by the compiler). Of course,
the use of generics to manage dependencies on language intrinsic types
remains unaffected by this recommendation.

\section{Comparison to J3's generics proposal for Fortran~202y}

{\sf Feel free to add a corresponding code version here,
since I am not sufficiently familiar with their approach.}


\bibliographystyle{plain}
\bibliography{traits}

\end{document}
