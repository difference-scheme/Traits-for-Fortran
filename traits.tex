\documentclass[11pt,oneside]{article}
\usepackage[a4paper, total={6.5in, 9in}]{geometry}
\usepackage{natbib}
\usepackage{url}
\usepackage{listings}
\usepackage{listings-rust}
\usepackage[scaled=0.82]{beramono}
\usepackage[T1]{fontenc}
\usepackage{hyperref}
\usepackage{xcolor}
\usepackage[title]{appendix}
\usepackage{csquotes}
\usepackage{mathpazo}
%\usepackage{mathptmx}

\newcommand{\code}[1]{{\selectfont\ttfamily{#1}}}

\lstdefinelanguage{LFortran}[]{Fortran}{
  morekeywords={abstract,type}
}

\frenchspacing

\begin{document}

\title{A trait system for the uniform expression of run-time
       and compile-time polymorphism in Fortran}

\author{Konstantinos Kifonidis, Ondrej Certik, Derick Carnazzola}

\maketitle

\abstract{.}

\section{Introduction}

Polymorphism was discovered in the 1960s by Kristen Nygaard and
Ole-Johan Dahl during their development of Simula~67, the world's
first object-oriented (OO) language \cite{Dahl_01}. Their work
introduced into programming what is nowadays known as ``virtual
method'' (i.e. function-pointer) based run-time polymorphism, which is
both the first focus of this document, and the defining feature of all
OO languages. Several other forms of polymorphism are known today, the
most important of them being parametric polymorphism (also known as
``generics''), which is the second focus of this document, and which
has historically developed disjointly from run-time polymorphism,
since it makes use of compile-time mechanisms.


\subsection{The purpose of polymorphism}

But what is the purpose of polymorphism in a programming language?
What is polymorphism actually good for? One of the more comprehensive
answers to this question was given by Robert C. Martin in numerous
books, as well as in the following quotation from his blog:

\begin{displayquote}
``There really is only one benefit to polymorphism; but it's a big
  one. It is the inversion of source code and run time
  dependencies. In most software systems when one function calls
  another, the runtime dependency and the source code dependency point
  in the same direction. The calling module depends on the called
  module. However, when polymorphism is injected between the two there
  is an inversion of the source code dependency. The calling module
  still depends on the called module at run time. However, the source
  code of the calling module does not depend upon the source code of
  the called module. Rather both modules depend upon a polymorphic
  interface. This inversion allows the called module to act like a
  plugin. Indeed, this is how all plugins work.''
\end{displayquote}

Notice, the absence of the words ``code reuse'' in these statements.
The purpose of polymorphism, according to Martin, is the ``inversion''
(i.e. replacement, or management) of source code dependencies by means
of particular abstractions, i.e. polymorphic interfaces (or
protocols/traits, as they are also known today). The possibility to
reuse code is then merely the logical consequence of such proper
dependency management.

\subsection{Source code dependencies}

Which then are the source code dependencies that polymorphism helps us
manage? It has been customary to make the following distinction when
answering this question:
\begin{itemize}
\item
 Firstly, most larger programs have dependencies on
 \emph{user-defined} procedures and data types. If the programmer
 employs encapsulation of not just a program's data, i.e. state, but
 also its procedures, both these dependencies can actually be viewed
 as dependencies on \emph{user-defined} abstract data types,
 i.e. types that contain both user-defined state, and implementation
 code which operates on that state. These are the dependencies that
 Martin is concretely referring to in the above quotation, and it is
 these dependencies on (volatile) implementation (details) that are
 particularly troublesome, because they lead to rigid coupling between
 the various different \emph{parts} of an application. Their results
 are recompilation cascades, the non-reusability of higher-level
 modules, the impossibility to comprehend a large application
 incrementally, and fragility of such an application as a whole.
\item
 Secondly, every program also has dependencies on abstract data types
 that are provided by the programming language in which it is written.
 Fortran's \code{integer}, \code{real}, etc. intrinsic types are
 examples of language intrinsic abstract data types. While hard-wired
 dependencies on such intrinsic types may not couple different parts
 of a program (because the implementations of these types are supplied
 by the language), they nevertheless make a program's source code
 rigid with respect to the data that it can be used on.
\end{itemize}

The most widely used approaches to manage dependencies on language
intrinsic types have so far been through generics, while dependency
management of user-defined types has so far been the task of OO
programming and OO design patterns. Martin has, for instance, defined
object-orientation as follows:

\begin{displayquote}
  ``OO is the ability, through the use of polymorphism, to gain
  absolute control over every source code dependency in [a software]
  system. It allows the architect to create a plugin architecture, in
  which modules that contain high-level policies are independent of
  modules that contain low-level details. The low-level details are
  relegated to plugin modules that can be deployed and developed
  independently from the modules that contain high-level policies.''
\end{displayquote}

\subsection{Modern developments}

Notice how Martin's modern definition of object-orientation, that
emphasizes source code decoupling, is the antithesis to the usually
taught ``OO'' approaches of one class rigidly inheriting
implementation code from another. Notice also how his definition does
not require some specific type of polymorphism for the task of
dependency management, as long as (according to Martin's first
quotation) the mechanism is based on polymorphic interfaces.

Martin's statements on the purpose of both polymorphism and OO simply
reflect the two crucial developments that have taken place in these
fields over the last decades. Namely, the realizations that
\begin{itemize}
\item
  run-time polymorphism should be freed from the conflicting concept
  of implementation inheritance (to which it was originally bound
  given its Simula~67 heritage), and be formulated exclusively in
  terms of conformance to polymorphic interfaces, i.e. function
  signatures, or purely procedural abstractions, and that
\item
  compile-time polymorphism should be formulated in exactly the same
  way as well.
\end{itemize}

Both these developments taken together have recently opened up the
possibility to treat polymorphism uniformly in a programming
language. As a consequence, it has become possible to use the
potentially more efficient (but also somewhat less flexible) mechanism
of compile-time polymorphism also for a number of tasks that have
traditionally been reserved for run-time polymorphism (i.e. OO
programming), and to mix and match the two polymorphism types inside a
single application to better satisfy a user's needs for both
efficiency and flexibility.

\subsection{Historical background}

The road towards these realizations has been surprisingly long. Over
the last five decades, a huge body of OO programming experience first
had to demonstrate that the use of (both single and multiple)
implementation inheritance breaks encapsulation in OO languages, and
therefore results in extremely tightly coupled, rigid, fragile, and
non-reusable code. This led to an entire specialized literature on OO
design patterns, that aimed at avoiding or mitigating the effects of
such rigidity, by replacing the use of implementation inheritance with
the means to formulate run-time polymorphism that are discussed
below. It also led to the apprehension that implementation inheritance
(but \emph{not} run-time polymorphism) should be abandoned. In modern
languages, implementation inheritance is either kept solely for
backwards compatibility reasons (e.g. in the Swift language), or it is
foregone altogether (e.g. in Rust, Go, and Carbon).

The first statically typed mainstream programming language that
offered a proper separation of run-time polymorphism from
implementation inheritance was Objective~C. It introduced
``protocols'' (i.e. polymorphic interfaces) in the year
1990. Protocols in Objective C consist of pure function signatures,
that lack implementation code. Objective~C provided a mechanism to
implement (multiple) such protocols by a class, and to thus make
classes conform to protocols. This can be viewed as a restricted form
of (multiple) inheritance, namely inheritance of object specification,
which is also known as \emph{subtyping}. Only a few years later, in
1995, the Java language hugely popularized these ideas under the
monikers ``interfaces'' and ``interface inheritance''. Today, nearly
all modern languages support polymorphic interfaces/protocols/traits,
and the mechanism of multiple interface inheritance that was
introduced to express run-time polymorphism in Objective~C. The only
negative exceptions in this respect being modern Fortran, and C++,
which both still stick to the obsolescent Simula~67 paradigm.

A similar learning process, as that outlined for run-time
polymorphism, took place in the field of compile-time/parametric
polymorphism. Early attempts, notably templates in C++, to render
function arguments and class parameters polymorphic, did not impose
any constraints on such arguments and parameters, that could be
checked by C++ compilers. With the known results on compilation times
and cryptical compiler error messages. Surprisingly, Java, the
language that truly popularized polymorphic interfaces in OOP, did not
provide an interface based mechanism to constrain its generics. Within
the pool of mainstream programming languages, this latter realization
was first made with the advent of Rust. Rust came with a trait
(i.e. polymorphic interface) system with which it is possible for the
user to uniformly and transparently express both generics
(i.e. compile-time) and run-time polymorphism in the same application,
and to relatively easily switch between the two, where possible.

Rust's main idea was quickly absorbed by almost all other mainstream
modern languages, most notably Go, Swift, and Carbon, with the
difference that these latter languages tend to leave the choice
between static and dynamic procedure dispatch to the compiler, or
language implementation, rather than the programmer. C++ is in the
process of adopting such constraints for its ``templates'' under the
term ``strong concepts'', but without implementing the greater idea to
uniformly express \emph{all} the polymorphism in the language through
them. An implementation of this latter idea must today be viewed as a
prerequisite in order to call a language design ``modern''. The
purpose of this document is to describe additions to Fortran, that aim
to provide the Fortran language with such modern capabilities.

\newpage

\section{Case Study: Calculating the average value of a numeric array}

To illustrate the advanced features and capabilities of some of the
available modern programming languages with respect to polymorphism,
and hence dependency management, we will make use here of a case
study: the simple test case of calculating the average value of a set
of numbers stored inside a one-dimensional array. In the remainder of
this section we will first provide an account and some straightforward
monomorphic (i.e. coupled) functional implementation of this test
problem, followed by a functional implementation that makes use of
both run-time and compile-time polymorphism. In the survey of
programming languages presented in Sect.~\ref{}, we will then recode
this test problem in an encapsulated fashion to highlight how the
source code dependencies in this problem can be managed in different
languages in more complex situations that require OO techniques in
order to hide state.

\subsection{Monomorphic functional implementation}

We have chosen Go here as a language to illustrate the basic ideas.
Go is easily understood, even by beginners, and is therefore well
suited for this purpose (another good choice would have been the Swift
language). The code in the following Listing~\ref{lst:procGo} should
be self explanatory for anyone who is even only remotely familiar with
the syntax of C family languages. So, we'll make only a few remarks
regarding the syntax.
\begin{itemize}
\item
  While mostly following a C like syntax, variable declarations in Go
  are essentially imitating Pascal syntax, where a variable's name
  precedes the declaration of the type.
\item
  Go has two forms of assignment. One for ``variables'' that are
  intended to be immutable, and one for regular mutable variables.
\item
  Go has array slices that most closely resemble those of Python's
  Numpy (which exclude the last element of an array slice).
\end{itemize}

Our algorithm for caclulating the average value of an array of integer
elements employs two different implementations for averaging. The first
makes use of a ``simple'' summation of all the array's elements, in
ascending order of their array index. While the second sums in a
``pairwise'' manner, dividing the array in half to carry out the
summations recursively, and switching to the ``simple'' method once
subdivision is no longer possible.

As a result, this code has some hard-wired dependencies on specific
user-defined implementations. Namely, function \code{pairwise\_sum}
depends on the implementation of function \code{simple\_sum}.
While functions \code{simple\_average} and \code{pairwise\_average}
depend on the implementations of functions \code{simple\_sum}, and
\code{pairwise\_sum}, respectively. In addition to these dependencies
on user-defined implementations, the entire program depends rigidly on
the \code{int} data type to declare the arrays that it is operating on,
as well as the results of the summations. Hence, this code cannot be
used on arrays of any other data type than \code{int}s.

\lstinputlisting[language=Go,style=boxed,label={lst:procGo},caption={Monomorphic functional version of the array averaging example in Go.}]{Code/Go/coupled.go}

\subsection{Polymorphic functional implementation}

{\sf To be inserted here \dots}

\newpage

\section{Survey of modern languages}

In the present section we will give implementations, in various modern
languages, of encapsulated (i.e. OO) code versions of the test
example. We will employ run-time polymorphism to manage the
dependencies on user-defined abstract data types, and generics in
order to manage the dependencies on language intrinsic abstract data
types. This will serve to illustrate how both run-time and
compile-time polymorphism can be typically used for dependency
management in these modern languages. The survey will also serve to
highlight the many commonalities but also some of the minor
differences in the approaches to polymorphism that were taken in these
different languages. As a final disclaimer: We do not generally
advocate to code problems, that are as simple as this test case, in
such an encapsulated manner. However, in more complex cases, where
state would be involved that would have to be hidden, an encapsulated
programming style will be difficult to avoid. Hence the test problem
will also have to stand in, in this section, for emulating such a more
complex problem, that would benefit from an encapsulated coding style.


\subsection{Go}

Go supported run-time polymorphism through (polymorphic)
``interfaces'' since its inception. Since version 1.18, Go supports
also compile-time polymorphism through generics that are bounded by
generics constraints. Such constraints in Go are expressed through
interfaces. Go therefore supports ``strong concepts''.

Go makes it explicit in its syntax that interfaces are types in their
own right, and that hence polymorphic variables (i.e. objects) can be
declared in terms of them. Encapsulation in Go is done by storing
state in a ``\code{struct}'' and by binding procedures that need to
use that state to this same \code{struct}, thus creating a
user-defined abstract data type (or ADT). Go allows the programmer to
make such ADTs conform to interfaces, by implementing all the
functions whose signatures are contained in an interface. There' no
explicit statement to implement interfaces. Instead an ADT is
implicitly assumed to implement an interface whenever it provides
implementations of all the interface's functions.

\subsubsection{Encapsulated version coded in Go}

Listing~\ref{lst:OOGo} gives an encapsulated version of the test
example coded in Go. The various versions of the \code{sum} functions
have been encapsulated in ADTs as described above. The crucial
elements for managing any dependencies on these user-defined ADTs in
this encapsulated code version are the (polymorphic) interfaces
\code{ISum} and \code{IAverager}. These are required in order to
express polymorphism for the various objects of the Sum and Averager
ADTs.

\lstinputlisting[language=Go,style=boxed,label={lst:OOGo},caption={Encapsulated Go version of the array averaging example.}]{Code/Go/mixed.go}


\subsection{Rust}

Rust supports both run-time and compile-time polymorphism through
polymorphic interfaces, which Rust calls ``traits''.

\subsubsection{Encapsulated version coded in Rust}

\lstinputlisting[language=Rust,style=boxed,label={lst:OORust},caption={Encapsulated Rust version of the array averaging example.}]{Code/Rust/mixed_poly/src/main.rs}

\subsection{Swift}

Being a successor language to Objective~C, Swift differs somewhat from
the languages considered so far in that it opted to retain
implementation inheritance for backwards compatibility to Objective~C,
whereas both Go and Rust do not support implementation inheritance
\emph{by design}. Swift therefore supports ``classical'' classes, but
it also allows one to bind methods to structures (which in contrast to
classes are value types in Swift).

Like the other modern languages, Swift supports both run-time and
compile-time polymorphism through polymorphic interfaces, that are
called ``protocols'' in Swift. If the Swift programmer chooses to
ignore implementation inheritance and classes, he can therefore very
much program with structures and protocols in Swift as he would with
structures and interfaces/traits in Go and Rust, respectively. We will
demonstrate this in the next section.

Swift has generics capabilities that are on par with those of Go and
Rust. For instance, all these languages allow for the parametrization
of structures (and in Swift's case also for classes). Where Swift
differs from both Go an Rust, is that it allows for parameterized
\emph{methods}. Whereas both Go and Rust have their users parameterize
entire interfaces/traits instead, in order to achieve the same
functionality.

The Swift way of directly parametrizing methods but \emph{not}
protocols has some advantages from the user's perspective, because it
almost completely obviates the need for manual instantiations of
generics. In particular, there is no need (compared to the other
languages) to always manually instantiate a different protocol for
every different generic type parameter that the user wishes to employ
with a generic method.

The Swift compiler is able to instantiate Swift generics largely
automatically, through inspection of the argument types that are
passed to functions and methods. In Swift, the user basically never
has to bother with instantiating a generic function, a method, and
often not even a structure or a class.

If the Swift programmer knows how to write generic functions, his
knowledge also automatically translates into coding generic methods,
since generic functions can be transformed into generic methods
without requiring any changes to their function signatures. This
property is helpful for the refactoring of non-OO codes into
corresponding OO versions.

\subsubsection{Encapsulated version coded in Swift}

Listing~\ref{lst:OOSwift} gives an example of how the encapsulated
version of the array averaging test problem can be programmed in
Swift. See the following remarks in order to understand this code:

\begin{itemize}
  \item
    Swift uses angled brackets \code{<>} to indicate generic parameter
    lists.
  \item
    Type constraints are formulated by supplying a protocol name after a
    type parameter (separated by a colon). Swift therefore supports
    ``strong concepts'', like the other languages considered here.
  \item
    Swift does not supply an equivalent to Go's \code{int | float64}
    syntax. Hence the user must use a \code{Numeric} protocol defined
    by the standard library, as a constraint for numeric types. Which
    leads to reliance on library code.
  \item
    Unfortunately, Swift's \code{Numeric} protocol does \emph{not} support
    the division operation! Hence the division that would have been required
    in function \code{average} of the \code{Averager} struct had to be moved
    out to the calling code of the main program.
  \item
    Interface (i.e. protocol) inheritance in Swift is signified with the
    \code{:} operator appearing behind a a structure's (or a class's)
    name. There is no separate \code{impl} block as in Rust, or implicit
    conformance to a protocol by implementing all its methods, as in Go.
  \item
    The Swift version makes use of language supplied structure constructors.
  \item
    Array slices are not arrays themselves. So an explicit conversion
    using an \code{Array()} constructor is required in such cases.
  \item
    Function and method calls in Swift make use of (mandatory) keyword
    arguments.
  \item
    The syntax for type conversion into a generic type \code{T} is
    somewhat peculiar. E.g. Go's \code{T(0)} is written as
    \code{T(exactly:0)!} in Swift (making use of the mandatory keyword
    ``\code{exactly}'' in the function responsible for the type
    conversion).
  \item
    Notice, how in contrast to both Go and Rust, the Swift code doesn't
    instantiate any interfaces by making use of type parameters for them.
    E.g. there is only a single version of the \code{IAverager} interface,
    and hence there's only a need in the main program to declare a single
    variable, \code{av}, of that interface (\code{av} being polymorphic
    over different \code{Averager} types). This single object is then
    straightforwardly used with data of \emph{both} the \code{Int} and
    \code{Float64} types.
\end{itemize}

\lstinputlisting[language=Swift,style=boxed,label={lst:OOSwift},caption={Encapsulated Swift version of the array averaging example.}]{Code/Swift/mixed.swift}

\subsection{Conclusions from the different implementations}

Since the use of run-time polymorphism by means of interfaces is very
similar in all languages considered here, we will only make some final
comments on the languages' compile-time, i.e.  generics features.

\subsubsection{Conclusions on Go}

Go's basic model to implement generics allows for parametrization of
structures, interfaces, and ordinary functions, but not methods.
Nevertheless, generic Go code is quite easy to read and to understand.
What sets Go apart from the other languages is its brilliant new
notion to interpret interfaces as type sets, and its syntax to support
this notion. This enables the Go programmer to easily tailor
constraints on generic types to his specific needs, which is what
makes the use of generics in Go very pleasant.

\subsubsection{Conclusions on Rust}

Rust's basic model to implement generics is similar to Go's in that it
allows for parametrization of structures, interfaces, and ordinary
functions. Being the oldest of the ``modern'' languages, Rust has a
few quirks which render its use for the management of all types of
dependencies through polymorphism somewhat suboptimal when compared to
the other languages considered here. Its basic deficiency is its
C++-like philosophy to copiously rely on external dependencies, even
for rather basic tasks, like initializing a generic type. The Rust
version of our test case is therefore marred by some avoidable
dependencies on external libraries (called ``crates'' in Rust), which
partly defeat the purpose of programming in a polymorphic fashion, in
order to avoid rigid dependencies. But even with the functionality
provided by such external dependencies, Rust doesn't allow type
conversion to generic types within generic routines. A necessary
capability for numerical work that is, for instance, built into Go.

\subsubsection{Conclusions on Swift}

We consider Swift's basic model of implementing generics by allowing
parametrized structures and methods (but not parametrized interfaces)
to be both the easiest to read, and the easiest to use from a
programmer's perspective. We hence consider it to be the most
attractive model to base Fortran's basic generic capabilities on,
provided that it can be implemented sufficiently easily, and that it
be enhanced by Go's idea to interpret interfaces as type sets.



\newpage

\section{New additions to Fortran: Subtyping}

\subsection{Named abstract interfaces (traits)}

\subsection{Multiple interface inheritance}

\subsection{Enhanced type declarations}

Named abstract interfaces are types in their own right. Hence,
(polymorphic) instance variables can be declared in terms of them. To
enable the Fortran programmer to do so, Fortran's \code{type}
declaration specifier needs to be extended to accept named abstract
interfaces.

\subsection{Improved structure constructors}

\subsection{File extension with predefined new defaults}

\newpage

\section{New additions to Fortran: Generics}

The new features that were discussed in the previous section are
required in order to uniformly express and support both run-time and
compile-time polymorphism in Fortran. We will now proceed with
discussing those additional new features that are exclusively required
in order to further support compile-time polymorphism, i.e. generics.

\subsection{Interfaces containing generic procedure signatures}
\label{sect:generic_interfaces}

Abstract interfaces should be allowed to contain signatures of generic
procedures, as in Swift. Go's and Rust's approach to parameterize
abstract interfaces themselves, appears not as attractive from a
user's perspective. The following code shows, as an example, an
abstract interface named \code{ISum} that contains the signature of a
generic type-bound procedure named \code{sum}:
\begin{lstlisting}[language=LFortran,style=boxed]
   abstract interface :: ISum
      function sum{INumeric :: T}(self,x) result(s)
         type(ISum), intent(in) :: self
         type(T),    intent(in) :: x(:)
         type(T)                :: s
      end function sum
   end interface
\end{lstlisting}

The example illustrates the use of a generic type parameter, i.e.  a
metatype, or a \emph{type of types}. In this example, this type
parameter is simply called \code{T}, and it is preceded by the name of
an abstract interface that expresses a constraint on the type
parameter. Fortran generics thus support ``strong concepts''. Both,
the type parameter and its constraint, are part of a generic type
parameter list that is enclosed in curly braces, and follows
immediately behind the procedure's name.


Notice that, since \code{T} is a metatype, there are some significant
differences to types that are specified in the standard parameter
list. For instance, the specification of a rank, or an \code{intent},
for metatypes like \code{T}, makes no sense. This is because the
latter are always scalar input parameters. The syntax used above,
that deviates slightly from how Fortran's usual function arguments are
declared, therefore appears justified as it reflects that, in type
parameters, one is dealing with different entities.

\subsection{Interfaces as type sets}

\begin{lstlisting}[language=LFortran,style=boxed]
   abstract interface :: INumeric
      integer | real(real64)
   end interface
\end{lstlisting}

For simple use cases, it should be optionally possible for the
programmer to employ a shorthand notation like in the following
modification of the example of an abstract interface declaration
given previously in Sect.~\ref{sect:generic_interfaces}:
\begin{lstlisting}[language=LFortran,style=boxed]
   abstract interface :: ISum
      function sum{integer | real(real64) :: T}(self,x) result(s)
         type(ISum), intent(in) :: self
         type(T),    intent(in) :: x(:)
         type(T)                :: s
      end function sum
   end interface
\end{lstlisting}
This would enable one to declare a type constraint for a generic type,
without having to explicitly declare an abstract interface for it
beforehand. The above notation would then define an abstract interface
implicitly, to be used as a type constraint for type \code{T}. In this
particular example, to admit only the default \code{integer}, or
\code{real(real64)} types, for \code{T}.

\subsection{Predefined interfaces for expressing common constraints}

The language should ideally supply some predefined, commonly used generic
constraints in the form of abstract interfaces that are contained in a
language intrinsic module. The actual implementation of these interfaces
could then, of course, employ the ``interfaces-as-type-sets'' syntax that
was described above. For instance, a more general \code{INumeric}
interface than the one given above, could be implemented as follows:
\begin{lstlisting}[language=LFortran,style=boxed]
   abstract interface :: INumeric
      integer(*) | real(*) | complex(*)
   end interface
\end{lstlisting}
Notice how this makes use of kind parameters to include all
\code{integer}, \code{real}, and \code{complex} types, admitted by the
language, in a single \code{abstract interface} constraint. The thus
defined, language provided, interface \code{INumeric} could then be
used from user code through a \code{use} statement like in the
following example
\begin{lstlisting}[language=LFortran,style=boxed]
module user_code

   use, intrinsic :: generic_constraints, only: INumeric

   abstract interface :: ISum
      function sum{INumeric :: T}(self,x) result(s)
         type(ISum), intent(in) :: self
         type(T),    intent(in) :: x(:)
         type(T)                :: s
      end function sum
   end interface

end module user_code
\end{lstlisting}
for use as a constraint in function and derived type implementations,
or in other interfaces, like \code{ISum} here, that require the
functionality of \code{INumeric}.

\subsection{Conversions to generic types}

\subsection{Generic type parameters for methods and procedures}

As already mentioned above, LFortran's design of generics should
follow Swift's, if possible, and allow generic type parameters
to be used in both type-bound (i.e. method) and ordinary
procedures. An implementation of the generic method \code{sum} of
Sect.~\ref{sect:generic_interfaces}, that is bound to a derived type
with name \code{SimpleSum}, would look as follows:
\begin{lstlisting}[language=LFortran,style=boxed]
   function sum{INumeric :: T}(self,x) result(s)
      type(SimpleSum), intent(in) :: self
      type(T),         intent(in) :: x(:)
      type(T)                     :: s
      integer :: i
      s = T(0)
      do i = 1, size(x)
         s = s + x(i)
      end do
   end function sum
\end{lstlisting}
While the next example illustrates how the same procedure would look
as a stand-alone (i.e. non-encapsulated) generic function:
\begin{lstlisting}[language=LFortran,style=boxed]
   function sum{INumeric :: T}(x) result(s)
      type(T), intent(in) :: x(:)
      type(T)             :: s
      integer :: i
      s = T(0)
      do i = 1, size(x)
         s = s + x(i)
      end do
   end function sum
\end{lstlisting}

\subsection{Generic type parameters for derived types}

In addition to procedures, generic type parameter lists must be allowed
also for derived types, as in the following example in which 
the interface \code{ISum} from above is implemented by a derived-type
named PairwiseSum:
\begin{lstlisting}[language=LFortran,style=boxed]
   type, implements(ISum) :: PairwiseSum{ISum :: U}
      private
      type(U) :: other
   contains
      procedure :: sum
   end type PairwiseSum
\end{lstlisting}
\code{PairwiseSum} depends on a generic type parameter \code{U}, that
is used in order to declare a field variable of \code{type(U)} within
\code{PairwiseSum}, that is named \code{other}. As is indicated by the
type constraint on \code{U}, object \code{other} conforms to the
\code{ISum} interface itself, and therefore contains its own
implementation of the \code{sum} procedure.


\subsection{Generic type parameters for structure constructors}

If a derived type is parameterized with a generic type, then its
structure constructor must also be assumed to be parameterized with
the same generic type. Hence, calls of structure constructors that are
instantiated with particular argument types replacing the generic type
parameters of their derived types, like e.g.
\begin{lstlisting}[language=LFortran,style=boxed]
   Averager{SimpleSum}()
   Averager{PairwiseSum{SimpleSum}}()
\end{lstlisting}
must be legal. Here, \code{SimpleSum} would be a derived type that
implements the \code{ISum} interface, but (in contrast to the
\code{PairwiseSum} type) is not parameterized by any generic type
parameters itself.

\newpage

\section{Proposed Fortran versions of the test example}

\subsection{Functional version}

{\sf To be inserted here \dots}

\subsection{Encapsulated version}

Listing~\ref{lst:OOFortran} gives our Fortran version of the
encapsulated form of the test example that corresponds to the code
versions that were presented in Sect.~\ref{} for all the other
languages.

\begin{itemize}
\item
  We employ here the Go borrowed syntax \code{integer | real(real64)}
  to implement the interface \code{INumeric}, that is used in order to
  express type genericity for the array \code{x} and the result of the
  summation \code{s} in our different implementations of method
  \code{sum}.
\item
  As in the corresponding Go version, \code{INumeric} is defined by
  the user himself as a type set consisting of the set of intersecting
  operations defined in Fortran for the \code{integer} and
  \code{real(real64)} types.  There is thus no need for an external
  dependency.
\item
  The remaining interfaces \code{ISum} and \code{IAverager} make use
  of generic methods that are declared in terms of \code{INumeric}.
  However, in contrast to the Go version, none of these interfaces is
  parameterized itself, since we followed Swift's model of generics.
\item
  Interface inheritance is expressed through the presence of the
  \code{implements(...)} specifier in a derived-type (i.e. class)
  definition (equivalent to Swift).
\item
  Conversions to generic types are done as in Go. Notice, how the
  compiler will have to do the necessary replacements of, e.g.,
  \code{T(0)} in function \code{sum} of class \code{SimpleSum} by
  calls to Fortran's correct conversion functions for integer and real
  types of the right kinds.
\item
  The example code makes use, in the main program, of the new
  structure constructors, with their enhancements that were discussed
  in Sect.~\ref{}, for the classes \code{Averager}, \code{SimpleSum},
  and \code{PairwiseSum}.
\item
  The Fortran version makes use of modules and \code{use} statements
  with \code{only} clauses, in order to make explicit the source code
  dependencies of the different defined classes.
\end{itemize}

\lstinputlisting[language=LFortran,style=boxed,label={lst:OOFortran},caption={Proposed encapsulated Fortran version of the array averaging example.}]{Code/Fortran/mixed.ft}

The most important point to notice in Listing~\ref{lst:OOFortran} is
how the main program is the only part of the code that (necessarily)
depends on implementations. The \emph{entire} rest of the code depends
merely on abstract interfaces (see the \code{use} statements in the
above modules). The Fortran version described here is therefore as
clean as the Go implementation with respect to dependency management,
and as easy to use as the Swift implementation.

The features described in this document have enabled us to avoid
rigidity in the program, by both decoupling it and making it operate
on multiple data types, thus allowing for a maximum of code
reuse. Notice the use of run-time polymorphism in
Listing~\ref{lst:OOFortran} by the \code{av} object of
\code{IAverager} type that is initialized in the \code{select case}
statement. This object necessarily cannot employ compile-time
polymorphism, as it is employed within a statement that performs a
run-time decision.

\subsection{Encapsulated (mostly) static version}

{\sf Do we need to have this one as a further example?}

\newpage

\section{Comparison to J3's generics proposal for Fortran~202y}

{\sf Feel free to add a corresponding code version here,
since I am not sufficiently familiar with their approach.}

\end{document}
